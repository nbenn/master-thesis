%% Memoir layout setup

%% NOTE: You are strongly advised not to change any of them unless you
%% know what you are doing.  These settings strongly interact in the
%% final look of the document.

%% Title page: redefine maketitle macro
\newlength{\logounitlength}
\setlength{\logounitlength}{0.08mm}

\def\thesistype#1{\def\thesistypestring{#1}}
\def\thesistypestring{}
\def\thesisperiod#1{\def\thesisperiodstring{#1}}
\def\thesisperiodstring{}
\def\thesistitle#1{\def\thesistitlestring{#1}}
\def\thesistitlestring{}
\def\thesisauthor#1{\def\thesisauthorstring{#1}}
\def\thesisauthorstring{}
\def\submissiondate#1{\def\submissiondatestring{#1}}
\def\submissiondatestring{}
\def\alternatereader#1{\def\alternatereaderstring{#1}}
\def\alternatereaderstring{}
\def\mainreader#1{\def\mainreaderstring{#1}}
\def\mainreaderstring{}

\newcommand{\ETHlogo}{
  \parbox{100mm}{
    \vbox{
      \kern2pt\setlength{\unitlength}{\logounitlength}
      \begin{picture}(330,110)(0,-5)
        \thicklines
        \multiput(0,0)(2,0){16}{\line(1,4){25}}
        \multiput(1,0)(0.5,2){12}{\line(1,0){84}}
        \multiput(42,40.4)(0.5,2){11}{\line(1,0){53}}
        \multiput(19.5,78.8)(0.5,2){12}{\line(1,0){210}}
        \multiput(116,0)(2,0){16}{\line(1,4){24}}
        \multiput(180,0)(2,0){16}{\line(1,4){25}}
        \multiput(237,0)(2,0){16}{\line(1,4){25}}
        \multiput(220,39)(0.5,2){12}{\line(1,0){30}}
        \put(262.5,100){\line(1,0){30}}
      \end{picture}
      \hfill\break\sffamily\bfseries Swiss Federal Institute of Technology 
      Zurich
    }
  }
}

\newcommand{\SfSlogo}{
  \parbox{40mm}{
    \hfill\kern2pt
    \setlength{\unitlength}{\logounitlength}\begin{picture}(330,110)(0,-5)
    \end{picture}\\
    \sffamily\bfseries
    \null\hfill Seminar for\kern3mm\break
    \null\hfill \phantom{g}Statistics\kern3mm%
  }
}

\makeatletter
\def\maketitle{
  \begingroup
  \hspace*{-3pt}\raise 20pt\hbox{\ETHlogo}\hfill
    \raise 19pt\hbox{\SfSlogo}\hspace*{-10pt}
  \linebreak
  \vspace{1pt}
  {\sffamily\bfseries\noindent{\\Departement of Mathematics}}
  \par\vspace*{5pt}
  \rule{\linewidth}{.3pt}\\[1pt]
  \vspace*{-15pt}
  \begin{center}
    \Large \thesistypestring
    \hfill
    \Large \thesisperiodstring\\[2pt] 
  \end{center}
  \rule{\linewidth}{.3pt}
  \medskip
  \vspace{70pt}
  \begin{center}
    \bfseries
    \Large \thesisauthorstring\\
    \vspace{2ex plus 1ex minus 1.5ex}
    \LARGE \thesistitlestring
  \end{center}
  \vspace{\stretch{2}}
  \vspace{\stretch{3}}

  \begin{center} \large
    \rule{.5\linewidth}{.3pt} \\[8pt]
    \begin{tabular}{ll}
      Submission Date: & \submissiondatestring
    \end{tabular}
    \\[5pt] \rule{.5\linewidth}{.3pt}
    
    \vspace{2ex plus 25ex minus 1.5ex}
    
    \begin{tabular}{ll}
      Co-Adviser: & \alternatereaderstring \\
      Adviser:    & \mainreaderstring
    \end{tabular}
  \end{center}
  \vspace*{-20pt}

  \endgroup
}
\makeatother

%% Turn extra space before chapter headings off.
\setlength{\beforechapskip}{0pt}

\nonzeroparskip
\parindent=0pt
\defaultlists

%% Chapter style redefinition
\makeatletter

\if@twoside
  \pagestyle{Ruled}
  \copypagestyle{chapter}{Ruled}
\else
  \pagestyle{ruled}
  \copypagestyle{chapter}{ruled}
\fi
\makeoddhead{chapter}{}{}{}
\makeevenhead{chapter}{}{}{}
\makeheadrule{chapter}{\textwidth}{0pt}
\copypagestyle{abstract}{empty}

\makechapterstyle{bianchimod}{%
  \chapterstyle{default}
  \renewcommand*{\chapnamefont}{\normalfont\Large\sffamily}
  \renewcommand*{\chapnumfont}{\normalfont\Large\sffamily}
  \renewcommand*{\printchaptername}{%
    \chapnamefont\centering\@chapapp}
  \renewcommand*{\printchapternum}{\chapnumfont {\thechapter}}
  \renewcommand*{\chaptitlefont}{\normalfont\huge\sffamily}
  \renewcommand*{\printchaptertitle}[1]{%
    \hrule\vskip\onelineskip \centering \chaptitlefont\textbf{\vphantom{gyM}##1}\par}
  \renewcommand*{\afterchaptertitle}{\vskip\onelineskip \hrule\vskip
    \afterchapskip}
  \renewcommand*{\printchapternonum}{%
    \vphantom{\chapnumfont {9}}\afterchapternum}}

%% Use the newly defined style
\chapterstyle{bianchimod}

\setsecheadstyle{\Large\bfseries\sffamily}
\setsubsecheadstyle{\large\bfseries\sffamily}
\setsubsubsecheadstyle{\bfseries\sffamily}
\setparaheadstyle{\normalsize\bfseries\sffamily}
\setsubparaheadstyle{\normalsize\itshape\sffamily}
\setsubparaindent{0pt}

%% Set captions to a more separated style for clearness
\captionnamefont{\sffamily\bfseries\footnotesize}
\captiontitlefont{\sffamily\footnotesize}
\setlength{\intextsep}{16pt}
\setlength{\belowcaptionskip}{1pt}

%% Set section and TOC numbering depth to subsection
\setsecnumdepth{subsection}
\settocdepth{subsection}

\checkandfixthelayout

\setlength{\droptitle}{-48pt}

%% don't insert spaces between chapters in lof/lot
\renewcommand{\insertchapterspace}{}

\setlength\cftbeforesectionskip{5pt}
\setlength\cftbeforesubsectionskip{2pt}

\setlength\cftbeforefigureskip{5pt}
\setlength\cftbeforetableskip{5pt}

\makeatother

%% This defines how theorems should look. Best leave as is.
\theoremstyle{plain}
\setlength\theorempostskipamount{0pt}

%% customized appearance for some characters
\renewcommand{\tilde}{\hbox{\raise.17ex\hbox{$\scriptstyle\mathtt{\sim}$}}}

%% fixed width columns in table with text justification
\newcolumntype{L}[1]{>{\raggedright\let\newline\\\arraybackslash\hspace{0pt}}m{#1}}
\newcolumntype{C}[1]{>{\centering\let\newline\\\arraybackslash\hspace{0pt}}m{#1}}
\newcolumntype{R}[1]{>{\raggedleft\let\newline\\\arraybackslash\hspace{0pt}}m{#1}}

%% code listings
\renewcommand{\theFancyVerbLine}{%
  \textcolor[rgb]{0.75,0.75,0.75}{\tiny\arabic{FancyVerbLine}}
}
\newminted[rcode]{r}{%
  tabsize=2,
  fontsize=\footnotesize,
  linenos,
  numbersep=4pt,
  style=knitr
}
\newtcolorbox[%
  auto counter,
  number within=chapter,
  list inside=codelist
]{rlisting}[4]{%
  list text={#2},
  title={\renewcommand{\baselinestretch}{1.00}\footnotesize\textbf{Listing \thetcbcounter}: #3},
  label={lst:#4},
  fonttitle=\sffamily,
  enhanced,
  frame hidden,
  interior hidden,
  sharp corners,
  boxrule=0pt,
  titlerule=0.5pt,
  bottomrule=0.5pt,%  <-- reserve space
  titlerule style=black,
  borderline south={0.5pt}{0pt}{black},%---- draw line
  colback=white,
  coltitle=black,
  boxsep=0mm,
  lefttitle=0cm,   % space to the left of title
  righttitle=0cm,  % space to the right of title
  bottomtitle=3mm, % space after title
  top=3mm,         % space at beginning of main
  bottom=3mm,      % space at end of main
  leftupper=6mm,   % space to the left of main
  breakable,
  #1
}
\definecolor{mintbg}{rgb}{0.95,0.95,0.95}
\newminted[rflow]{r}{%
  style=knitr,
  bgcolor=mintbg,
  frame=single,
  rulecolor=\color{mintbg},
  fontsize=\small,
  framesep=5px,
  tabsize=2
}
\newminted[bashflow]{bash}{%
  style=knitr,
  bgcolor=mintbg,
  frame=single,
  rulecolor=\color{mintbg},
  fontsize=\small,
  framesep=5px,
  tabsize=2
}
\setmintedinline[r]{style=knitr, breaklines}
\setmintedinline[bash]{style=knitr, breaklines}
\setmintedinline[text]{style=knitr, breaklines}
