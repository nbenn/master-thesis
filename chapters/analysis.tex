\chapter{Data Analysis}
\label{ch:data-analysis}

The initial goal of this project was concerned with modeling different infection patterns as influenced by \gls{sirna} mediated gene knockdown. By fitting a \gls{glm} model, using the large set of available cellular features, described in section \ref{sec:scf-data}, as predictor variables and the membership to one of two wells as binary response, it was conjectured to be possible to determine a subset of influential features. Unfortunately, to present data, this has not been possible.

In order the make the substantial amount of single cell data, obtained from several large-scale, high throughput \gls{sirna} screens performed by the InfectX consortium, available to an environment for statistical analysis, an R packages was developed, which is described in chapter \ref{ch:singlecellfeatures}. Building on this crucial piece of infrastructure, the current chapter will explore some of the data analysis that was performed, beginning a short introduction into the models used and continuing with preliminary findings that motivate the investigation of possible normalization schemes. A final outlook on possible improvements will conclude this chapter.

\section{Statistical Models}
Many algorithms for binary classification exist, including decision trees, \glspl{svm}, Bayesian networks, neural networks and \glspl{glm}, some of which have been encountered in previous sections due to their application in infection scoring (see section \ref{sec:infection-scoring}). As neither prediction not classification per se are of main interest, binary logistic regression presents an attractive method due to availability of extensive coefficient and model statistics. Therefore, modeling of \gls{sirna} effects on cellular features is performed, using the glm function provided by the R stats package \citep{RCoreTeam2015}, as well as glmnet belonging to the CRAN package of the same name \citep{Friedman2010a}.

A great many binary comparisons are possible with the given datasets. In order to focus on wells where there is reason to assume they might be biologically interesting, the $n \choose 2$ possible combinations of wells (within a single plate, \tilde 70000 pairs can be formed), are narrowed down using a \gls{pmm} as derived by \cite{Ramo2014}. Of the resulting hit list, several genes are selected and compared to wells containing scrambled \gls{sirna} reagents, which should provide a good choice for establishing background levels. A possible alternative are mock controls but owing to the complete absence of \gls{sirna} molecules, the difference in treatment of cells is only increased and hence it can be conjectured that they provide inferior comparisons.

\subsection{Generalized Linear Models}
Modeling the relationship among variables is one of the most important applications of statistical theory. The study of regression analysis (and the closely related notion of correlation) started to form towards the end of the 19th century with Sir Francis Galton's study of height heredity in humans and his observation of regression towards the mean. Over the next few years, Udny Yule and Karl Pearson cast the developed concepts into precise mathematical formulation, in turn building on work performed by Adrien-Marie Legendre and Carl Friedrich Gauss who developed the method of least squares almost a century earlier \citep{Allen1997}.

A multiple linear regression model can be written in matrix-vector form as
\begin{equation}
  y = X \beta + \epsilon \label{eq:lin-reg}
\end{equation}
where $y \in \R^n$ is the vector of observations on the dependent variable, the design matrix $X \in \R^{n \times p}$ contains data on the independent variables, $\beta \in \R^p$ is the $p$-dimensional parameter vector and the error term $\epsilon \in \R^n$ captures effects not modeled by the regressors.

In order to find unknown coefficients $\beta_i$, the ordinary least squares estimator minimizes the residual sum of squares, the squared differences between observed responses and their predictions according to the linear model.
\begin{subequations}
\begin{align}
  \Hbeta &= \argmin_{\beta} \norm{y - X \beta}^2 \\
         &= (X^T X)^{-1} X^T y \label{eq:ols-estimate}
\end{align}
\end{subequations}
Some assumptions are typically associated with linear regression models that yield desirable attributes for the estimates. None of these restrictions are imposed on the explanatory variables; they can be continuous or discrete and combined as well as transformed arbitrarily. Furthermore, in practice, it is irrelevant whether the covariates are treated as random variables or as deterministic constants. With exception of the field of econometrics it appears that the majority of literature adheres to the latter interpretation and therefore, statements will not explicitly be conditional on covariate values.
\begin{description}
  \item[Linearity.] The relationship between dependent and independent variables is assumed to be linear (after suitable transformations) and individual effects additive. If this cannot be satisfied, a linear model is not suitable.
  \item[Full rank.] For the matrix $X^T X$ to be invertible, it has to have full rank $p$. Therefore $n \leq p$ and all covariates must be linearly independent.
  \item[Exogeneity.] All independent variables should be known exactly i.e. contain no measurement or observation errors as only the mean squared error of the dependent variable is minimized. Additionally, all important causal factors have to be included in the model. Exogeneity implies $\Erw[\epsilon_i] = 0 \forall i$, as well no correlation between regressors and error terms \citep{Hayashi2000}.
  \item[Spherical errors.] This includes both homoscedasticity or constant error variance, $\Erw[\epsilon_i^2] = \sigma^2 \forall i$, and uncorrelated errors $\Erw[\epsilon_i \epsilon_j] = 0 \forall i \neq j$. These two conditions can be written more compactly as $\Var(\epsilon) = \sigma^2I_{n \times n}$.
  \item[Normality.] For the estimated coefficients to gain some additional desirable characteristics, it can be required that the errors $\epsilon_i$ be jointly normally distributed. Together with the above restrictions on expectation and variance, this yields $\epsilon \mathbin{\sim} \N_n(0, \sigma^2 I_{n \times n})$.
\end{description}

Violations of these assumptions have varying consequences. In situations of perfect collinearity, the ordinary least squares estimator $\Hbeta$ as defined in \eqref{eq:ols-estimate} does not exist. Recovering such a situation is possible by using a generalized matrix inverse (for example the Moore--Penrose pseudoinverse) or by dropping the corresponding variables. High correlation inflates coefficient variances, which may be countered by employing a regularization scheme like ridge regression. Omitting a variable that is both correlated with dependent variables and has an effect on the response (a nonzero true coefficient) will introduce bias in the parameters. The method of instrumental variables can help to produce an unbiased estimator.

The assumption of spherical errors ensures that the least squares estimator is the best linear unbiased estimator in the sense that it has minimal variance among all linear unbiased estimators. Heteroscedasticity and autocorrelation do not yield biased coefficient estimates but can introduce bias in \gls{ols} estimates of variance, causing inaccurate standard errors. Such a situation calls for generalized least squares estimation, for example weighted least squares when the data is not homoscedastic ($\sigma^2$ is vector-valued but off-diagonal elements of $\Var(\epsilon)$ are zero), or feasible generalized least squares which can be applied in case of heteroscedasticity and\slash or correlation between errors. Whenever the condition of spherical errors does not hold, \gls{ols} is inefficient and generalized least squares estimators have smaller variance.

Finally, normality provides the framework necessary for applying common hypothesis testing, yielding t-statistics, p-values and confidence intervals for coefficients, as well as an F-statistic for the model as a whole. Furthermore, under normality assumptions, the \gls{ols} estimator and \gls{mle} coincide. Quantile regression and other forms of robust regression, such as M estimators may restore validity of inference when errors do not follow a normal distribution.

Modeling data to a dichotomous response variable with \gls{ols} methodology, while readily possible, often is a bad choice due to violations of several of the above assumptions. First of all, normal distributions are continous with support $x \in \R$ and thusly a categorical variable cannot be normally distributed. Homoscedasticity does not hold as well as can easily be seen in a geometric argument: any line with nonzero slope, fitted to a set of $Y$ with $Y_i \in \{0,1\}$, will produce residuals that vary linearly. Finally and perhaps most importantly, a linear model is unsuitable due to there being no constraints on the range of fitted values. While this could be dealt with, the concept of additivity within this context raises questions of its own when fitted values are thought of as probabilities. Linear behavior throughout the range of 0 (impossible) to 1 (certain) makes no sense for most practical applications, as typically, some flattening is expected when approaching either end of the spectrum.

In view of the above considerations it becomes clear that some sort of extension to ordinary linear regression in needed in order to deal with binary response variables. A theory, unifying several previously separately treated statistical models, including linear regression, \gls{anova}, logistic regression, Poisson regression and multinomial response, as \acrfullpl{glm}, was developped by John Nelder and Robert Wedderburn in the early 1970's \citep{Nelder1972}. Much of this section is based on that work.

In order to accommodate the newly added extensions, the classical linear model is rewritten as

\begin{equation}
  \Erw[Y] = \mu \text{ where } \mu = X \beta,
\end{equation}

yielding a three-part specification, consisting of

\begin{enumerate}[label=(\alph*)]
  \item the \textit{random component}; $Y$ is distributed according to a member of the exponential family,\footnote{Many of the most commonly used distributions belong to the exponential family, including but not restricted to Bernoulli, binomial, Poisson, exponential and normal distributions. Common to all members, the probability density function can be written as
  \begin{equation}
    f(x;\theta) = h(x) e^{\theta^\intercal T(x)-A(\theta)}
  \end{equation}
  where $\theta$ is the vector of paramteters, $T(x)$ the vector of sufficient statistics, $A(\theta)$ the cumulant generating function and $h(x)$ the base measure. In case of the Bernoulli distribution
  \begin{equation}
    f(k;\pi) = \pi^k (1-\pi)^{1-k}\label{eq:bern-pmf}
  \end{equation}
  this gives $h(k) = 1$ $T(k) = k$, $\theta = \log\frac{\pi}{1-\pi}$ and $A(\theta) = \log(1+e^\theta)$. Restricting \glspl{glm} to exponential family distributions makes it possible to stay within the framework of maximum likelihood parameter estimations, as given this restriction, \gls{mle} yields the best parameter estimator with respect to minimal variance.}
  \item the \textit{systematic component}; a linear predictor is given by $\eta = X\beta$,
  \item and the \textit{link} between random and systematic components, expressed as the link function $g(\cdot)$, such that $\eta = g(\mu)$; in case of normally distributed Y, $\mu = \eta$ (identity link).
\end{enumerate}

\renewcommand{\arraystretch}{2}
\setlength{\tabcolsep}{0.2em}
\begin{table}
  \centering
  \caption[Link functions of common univariate distributions of the exponential family.]{Common univariate distributions of the exponential family alongside mean and canonical link functions.}
  \label{tab:glm-links}
  \footnotesize
  \begin{tabular}{L{0.18\linewidth}L{0.18\linewidth}L{0.18\linewidth}L{0.18\linewidth}L{0.18\linewidth}}
    Distribution &
      Support &
      Link name &
      Link function &
      Mean function \\
    \hline 
    Normal &
      $(-\infty, +\infty)$ &
      identity &
      $\eta = \mu$ &
      $\mu = X\beta$\\
    Poisson &
      $\{0, 1, 2, \dotsc\}$ &
      log &
      $\eta = \log(\mu)$ &
      $\mu = e^{X\beta}$\\
    Binomial &
      $\{0, 1, \dotsc, N\}$ &
      logit &
      $\eta = \log\left(\frac{\mu}{1-\mu}\right)$ &
      $\mu = \frac{e^{X\beta}}{1+e^{X\beta}}$\\
    Gamma &
      $(0, +\infty)$ &
      reciprocal &
      $\eta = -\mu^{-1}$ &
      $\mu = -(X\beta)^{-1}$\\
    Inverse Gaussian &
      $(0, +\infty)$ &
      inverse squared &
      $\eta = -\mu^{-2}$ &
      $\mu = (-2X\beta)^{-1/2}$\\
  \end{tabular}
\end{table}

Mean and canonical link functions for several common univariate distributions belonging to the exponential family are shown in table \ref{tab:glm-links}. Binary response can be written as 
\begin{equation}
  Y_i \mathbin{\sim} \Bern(\pi_i)
\end{equation}
which implies that
\begin{equation}
  \P(Y_i=1) = 1 - \P(Y_i=0) = \pi_i.
\end{equation}
Therefore, $\pi_i$ can be thought of as the probability of one outcome (i.e. success) and its complement ($1-\pi_i$) corresponds to the probability of the other outcome (failure). Each experimental unit that yields one outcome is associated with a vector of independent variables $(x_{i,1}, x_{i,2}, \dotsc, x_{i,p})$ and the goal of regression in this context becomes modeling the relationship between response probability $\pi_i$ and explanatory variables $(x_{i,1}, x_{i,2}, \dotsc, x_{i,p})$. A suitable link function for Bernoulli distributed response, as discussed above, is required to map the unit interval onto the whole real line and preferably be shaped sigmoidally in order accurately describe increasingly likely as well as increasingly unlikely events. Three possibilities are often considered within this context,

\begin{enumerate}[label=(\alph*)]
  \item the \textit{logit} or logistic function
  \begin{equation}
    g(\pi) = \log\left(\frac{\pi}{1-\pi}\right),
  \end{equation}
  \item the \textit{probit} or inverse normal function
  \begin{equation}
    g(\pi) = \Phi^{-1}(\pi),
  \end{equation}
  with $\Phi(\cdot)$ denoting the cumulative distribution function of the normal distribution,
  \item and the \textit{complementary log-log} function
  \begin{equation}
    g(\pi) = \log\left(-\log(1-\pi)\right).
  \end{equation}
\end{enumerate}

All three functions are continuous and increasing on $(0,1)$ and the first two are symmetric in the sense that $g(\pi) = -g(1-\pi)$, while the third is not. Using a logit link function, the probability of a positive response can therefore be written as
\begin{equation}
  \pi_i = \frac{e^{x_i^\intercal \beta}}{1+e^{x_i^\intercal \beta}}
\end{equation}
and the model is, slightly rearranged, stated as
\begin{equation}
  \log\left(\frac{\pi_i}{1-\pi_i}\right) = x_i^\intercal \beta = \sum_{j=1}^p x_{i,j}\beta_j.\label{eq:log-model}
\end{equation}

where $x_i$ is shorthand for the $i$-th row $(x_{i,1}, x_{i,2}, \dotsc, x_{i,p})$ of the design matrix $X$. Consequently, of a unit change in a covariate $x_{i,j}$ will increase the corresponding probability log-odds by a multiplicative factor $\exp(\beta_j)$, as can easily be seen by exponentiating expression \ref{eq:log-model}.

The likelihood of a set of parameter values $\pi$, given the data $y_i$, is equal to the probability of the data, given the parameters. Hence, in a logistic regression model (see expression \ref{eq:bern-pmf} for the probability mass function of a Bernoulli distribution), we have

\begin{subequations}
\begin{align}
  L(\beta; y) = \P(y; \beta) &= \prod_{i=1}^n \pi_i^{y_i} (1-\pi_i)^{1-y_i} \label{eq:lik-any} \\
  &= \prod_{i=1}^n \frac{e^{x_i^\intercal \beta y_i}}{1+e^{x_i^\intercal \beta}} \label{eq:lik-logit}
\end{align}
\end{subequations}

and expressed, for reasons of convenience, as log-likelihood

\begin{subequations}
\begin{align}
  l(\beta; y) &= \sum_{i=1}^n y_i \log(\pi_i) + (1-y_i) \log(1-\pi_i) \label{eq:loglik-any} \\
  &= \sum_{i=1}^n x_i^\intercal \beta y_i -  \log\left(1+e^{x_i^\intercal \beta}\right). \label{eq:loglik-logit}
\end{align}
\end{subequations}

Both expressions \ref{eq:lik-logit} and \ref{eq:loglik-logit} are derived using the logit link function but any other of the proposed links can be used to substitute $\pi_i$ in equations \ref{eq:lik-any} and \ref{eq:loglik-any}. In order to find the maximum likelihood estimates
\begin{equation}
\Hbeta = \argmax_{\beta} l(\beta; y), 
\end{equation}

the first derivative of $l(\beta; y)$ with respect to $\beta_j$ in required.

\begin{subequations}\label{eq:firstder}
\begin{align}
  \frac{\partial l(\beta)}{\partial \beta_j} &= \sum_{i=1}^n \frac{y_i - \pi_i}{\pi_i (1-\pi_i)} \frac{d \pi_i}{d \eta_i} \frac{\partial \eta_i}{\partial \beta_j} \label{eq:firstder-any} \\
  &= \sum_{i=1}^n \left(y_i - \frac{e^{x_i^\intercal \beta}}{1+e^{x_i^\intercal \beta}}\right) x_{i,j} \label{eq:firstder-logit}
\end{align}
\end{subequations}

Again, \ref{eq:firstder-logit} is obtained from \ref{eq:firstder-any} by using a logit link, and consequently substituting 

\begin{equation*}
  \frac{d \pi_i}{d \eta_i} = \frac{-e^{-\eta_i}}{(1+e^{-\eta_i})} \text{,\ \ } \eta_i = x_i^\intercal \beta \text{\ \ and\ \ } \frac{\partial \eta_i}{\partial \beta_j} = x_{i,j}.
\end{equation*}

The log-likelihood maximum is found by setting the first derivatives in \ref{eq:firstder} to zero and solving for $\beta$. No closed form solutions to the resulting equations exist and therefore numerical algorithms such as Newton-Raphson are usually used, which can be formulated as an iteratively re-weighted least squares regression problem. Newton's method attempts to find the root $x_r$ of a differentiable function $f(x)$ by starting with an initial guess $x_0$ and iterating

\begin{equation}
  x_{n+1} = x_n - \frac{f(x_n)}{f^\prime(x_n)}.
\end{equation}

When a good initial guess is made and some restrictions on $f(x)$ apply, the rate of convergence is quadratic\footnote{Given a function$f(x)$, that is three-times differentiable in an interval $I^\ast= [a,b]$ such that $a < x_r < b$ and $f^\prime(x_r) \neq 0$, there exists an interval $I = [x_r-\delta, x_r+\delta]$, with $\delta > 0$ for which every $x_0 \in I$ converges quadratically towards $x_r$ \citep{Schwarz2006}.} and even if certain conditions do not hold, or the starting point is not suitably chosen, the algorithm may still converge towards the root. Returning to maximum likelihood notation of expression \ref{eq:firstder}, a Newton-Raphson iteration can be written as

\begin{equation}
  \beta^{(n+1)} = \beta^{(n)} - \frac{l^\prime(\beta^{(n)})}{l^{\prime\prime}(\beta^{(n)})}.\label{eq:nr-loglik}
\end{equation}

The second derivative of the log likelihood with respect to $\beta$, as needed in \ref{eq:nr-loglik} is

\begin{equation}
  \frac{\partial^2 l(\beta)}{\partial \beta_j\partial \beta_k} = -\sum_{i=1}^n \frac{1}{\pi_i (1-\pi_i)} \left(\frac{d \pi_i}{d \eta_i}\right)^2 \frac{\partial \eta_i}{\partial \beta_j} \frac{\partial \eta_i}{\partial \beta_k},
\end{equation}

which, in matrix form, can be written as

\begin{equation}
   \frac{\partial^2 l(\beta)}{\partial \beta\partial \beta^\intercal} = -X^\intercal W X,
\end{equation}

with the $n \times n$ diagonal matrix $W$

\begin{equation*}
  W = \diag\left\{\frac{1}{\pi_i (1-\pi_i)} \left(\frac{d \pi_i}{d \eta_i}\right)^2\right\} \text{,\ \ as\ \ } \frac{\partial \eta_i}{\partial \beta_j} = x_{i,j} \text{\ \ and \ \ } \frac{\partial \eta_i}{\partial \beta_k} = x_{i,k}.
\end{equation*}

In the case of logistic regression, $W$ simplifies to $w = \diag\{\pi_i (1-\pi_i)\}$ and $l^\prime(\beta)$ can be written in matrix form as $l^\prime(\beta) = X^\intercal (y - \pi)$.

Using these results, expression \ref{eq:nr-loglik} can be rewritten as

\begin{equation}
  \beta^{(n+1)} = \beta^{(n)} + \frac{X^\intercal (y - \pi)}{X^\intercal W X},
\end{equation}

which is iterated until the updates become smaller than a threshold, at which point convergence is said to have been reached.

% finish with linking to irls (see esl, p120)

%\subsection{Regularization}
%Two regularization schemes, ridge regression and lasso regression, will be introduced very briefly due to their availability in glmnet.
% esl p61

\subsection{Parallel Mixed Model}
As a way of exploiting replication, typically involved in \gls{sirna} screening, at the level of individual sequences, different sequences targeting the same gene and as is the case for InfectX, different treatments in the form of varying pathogens, \citeauthor{Ramo2014} developed a \gls{pmm}, capable of gaining statistical power from structural parallelism. All following remarks are adapted from the referenced publication and the topic is only outlined briefly for completeness sake. The interested reader is directed to \cite{Ramo2014} for more information. The proposed model is written as

\begin{equation}
  y_{pgs} = \mu_p + a_g + b_{pg} + \epsilon_{pgs},
\end{equation}

where $y_{pgs}$ represents the per-well phenotype (e.g. infection score), which is described as the sum of a fixed effect $\mu_p$ for pathogen $p$, as well as random effects $a_g$ for gene $g$, $b_{pg}$, a correction for gene $g$ within pathogen $p$, and an error term $e_{pgs}$. The overall effect of gene knockdown $g$ under pathogen treatment $p$ therefore is 



\begin{knitrout}
\definecolor{shadecolor}{rgb}{0.969, 0.969, 0.969}\color{fgcolor}\begin{figure}

{\centering \includegraphics[width=.95\linewidth]{figures/R/pmm-pmm-heatmap-1} 

}

\caption[Heatmap plot showing significant genes for InfectX kinome screens as determined by pmm.]{A heatmap plot as produced by the Bioconductor package pmm, which displays all genes that were determined to be significant hits (FDR < 4; indicated by black borders) for at least one pathogen. Genes are ordered by average c\textunderscript{pg} values and both extrema are marked with white dots, while the sharedness score is shown as a scatterplot below. All available kinome screens were taken into consideration.}\label{fig:pmm-heatmap}
\end{figure}


\end{knitrout}




\begin{equation}
  c_{pg} = a_g + b_{pg},
\end{equation}

and a positive estimated effect corresponds to enhanced infection levels, while a negative $c_{pg}$ value indicates reduced infectivity. Random effects are distributed as $a_g \mathbin{\sim} \N(0, \sigma_a^2)$, $b_{pg} \mathbin{\sim} \N(0, \sigma_b^2)$ and $\epsilon_{pgs} \mathbin{\sim} \N(0, \sigma_\epsilon^2)$, while estimation is carried out by the CRAN package lme4, using restricted maximum likelihood. Hits are selected according to an estimated local \gls{fdr} which assumes that a mixture of two distributions corresponding to genes that are no hits ($f_0$) and genes that actually are hits ($f_1$), generates the overall distribution. Furthermore, it is assumed that $f_0 \mathbin{\sim} \N(0, \sigma_a^2+\sigma_b^2)$ and $f_0 \mathbin{\sim} \N(\theta, \sigma_a^2+\sigma_b^2)$, i.e. the two distributions are identical except for a shift in mean by $\theta$. The overall distribution can be expressed as $f(c_{pg}) = \pi_0 f_0(c_{pg}) + (1-\pi_0) f_1(c_{pg})$, where $\pi_0$ is the proportion of true hits. Finally, the \gls{fdr} is defined as

\begin{equation}
  q_{pg} = \fdr(c_{pg}) = \frac{\pi_0 f_0(c_{pg})}{f(c_{pg})}
\end{equation}

and represents the probability that the effect for a given gene and pathogen is a false discovery. Figure \ref{fig:pmm-heatmap} displays a visualization obtained by applying the PMM package (available on Bioconductor) to kinome-wide InfectX screens. Color-coding corresponds to estimated $c_{pg}$ effects and columns are sorted according to descending mean values. Only genes are included where the estimated \gls{fdr} is below 0.4 for at least one pathogen (indicated by black borders), suggesting that up to 40\% of individual hits may be superfluous. Centered white dots indicate the maximum and minimum $c_{pg}$ value for each pathogen. For each gene, a sharedness score is displayed as well. This quantity is defined as

\begin{equation}
  s_g = \frac{1}{2} \left(\left(1-\frac{1}{P}\sum_{p=1}^P q_{pg}\right) + \frac{\sum_{p=1}^P \mathds{1}_{q_{pg} < 1}}{P}\right)
\end{equation}

and quantifies how common a hit gene is among pathogens by quantifying both the extent of downward shift from 1 of the mean $q_pg$ value (over all $P$ pathogens), as well as the fraction of pathogens that contribute $q_{pg} < 1$ instances.

\section{Preliminary Findings}
Due to the high degree of redundancy in feature extraction during image analysis, a significant amount of correlation among features can be expected. Measurements of objects describing similar image segments, as well as features that build on related concepts, such as mean, median or integrated intensities, will obviously yield similar values and cause a dependence structure that has to be dealt with in statistical data analysis. Figure \ref{fig:analysis-correlation} visualizes the issue by showing a heatmap representation of a correlation matrix, obtained from all available \textit{AreaShape}, \textit{Intensity} and \textit{Texture} features for a \textit{Brucella} plate using Dharmacon \gls{sirna} (J110-2D) and a randomly sampled subpopulation of cells (10\%). The three feature groups can easily be spotted as diagonal blocks with high within group and lower between group correlation, where the situation is worst for intensity features due to the extensive set of measurements quantifying similar properties (see figure \ref{fig:intensity-features}).

\begin{knitrout}
\definecolor{shadecolor}{rgb}{0.969, 0.969, 0.969}\color{fgcolor}\begin{figure}

{\centering \includegraphics[width=\maxwidth]{figures/R/correlation-heatmap-analysis-correlation-1} 

}

\caption[Heatmap representation of correlation among single cell features.]{A heatmap representation of the correlation matrix obtained by sampling 10\% of single cell feature data available for plate J110-2D illustrates severe correlation among many features that is typical for all datasets. This comes as no surprise due to the redundancy in measured features. The three diagonal blocks correspond to three groups of features, \textit{AreaShape}, \textit{Intensity} and \textit{Texture}.}\label{fig:analysis-correlation}
\end{figure}


\end{knitrout}



In order to deal with this data issue, features were transformed to the coordinate system of \glspl{pc} prior to \gls{glm} model fitting. When employing \gls{pca}, typically the first 10\% of \glspl{pc} capture around 90\% of the overall variance in the data, corroborating the above claim of significant correlation among feature vectors. A further problem that is present in many datasets is that pairs of wells can be perfectly separated. This causes problems in maximum likelihood estimation, as affected coefficients are allowed to grow arbitrarily but is not unexpected given the large design space. \Gls{pca} provides a tool for addressing the issue by encouraging only inclusion of a subset of \glspl{pc} and thus reducing dimensionality.

Several \gls{glm} model fits, based on principal component regression are summarized in table \ref{tab:glm-1}. For the same plate that was used to generate figure \ref{fig:intensity-features} (J110-2D, \textit{Brucella}, Dharmacon unpooled, replicate 1), well pairs corresponding to the genes identified by \gls{pmm} as down-hits, MTOR (H6) and PIK3R3 (K8), as well as up-hits, RIPK4 (G17) and TGFBR1 (M4), with respect to infection, are formed with all available scrambled wells of the given plate. In order to establish a baseline of sorts, scrambled wells are also paired with each other.

In order to describe how well the discrepancy between the two input wells is captured, some model characteristics, alongside scores for out-of-sample predictions are reported. The AIC is a goodness of fit estimate, composed of the maximized likelihood $L$ and a penalty term for model size $k$ and is defined as

\begin{equation}
  \aic = 2k - 2l(\Hpi; y).
\end{equation}

Deviance values require caution when interpreted as a goodness-of-fit criterion, especially when used as an absolute measure, rather than being employed in a comparative capacity for nested models (analysis of deviance). The glm function of the R stats package reports both null deviance, defined as

\begin{equation}
  D^{(0)}(y; \pi^{(0)}) = 2l(\widetilde{\pi}; y) - 2l(\pi^{(0)}; y),
\end{equation}

with $\widetilde{\pi}$ denoting the saturated model, and residual deviance

\begin{equation}
  D(y; \Hpi) = 2l(\widetilde{\pi}; y) - 2l(\Hpi; y).
\end{equation}

The saturated model contains as many parameters as there are data observations ($n$) and consequently represents the maximally possible likelihood. The null model $\pi^{(0)}$ contains only an intercept term (i.e. a single parameter), while the proposed model $\Hpi$ attempts to explain the data using $p+1$ parameters (one for each covariate and an intercept). Finally, the values reported in table \ref{tab:glm-1} as $\textDelta\textsubscript{deviance}$ are obtained as

\begin{equation}
  \textDelta\textsubscript{deviance} = D^{(0)}(y; \pi^{(0)}) - D(y; \Hpi),
\end{equation}

therefore describing the difference between the quality of fit of the null model to that of the estimated model. Similarly, for the degrees of freedom, 

\begin{equation}
  \textDelta\textsubscript{df} = \text{df}_\text{null} - \text{df}_\text{res} = n-1 -(n-(p+1)) = p.
\end{equation}

Under certain assumptions,\footnote{Deviance is only distributed as $\chi^2$ in the limit where for each $i \in \{1, 2, \dotsc, n\}$, the number of identical covariate rows $x_i$ grows to infinity. For continuous regressors, this is typically not the case, as the number of unique $x_i$ will often be very close to $n$. In case of binomially distributed response, the possibility of over-dispersion causes additional issues, which are not further described, as this does not apply to the current situation. Nevertheless, \citeauthor{Nelder1972} state that ``[t]he $\chi^2$ approximation is usually quite accurate for differences of deviances even though it is inaccurate for the deviances themselves.''} $\textDelta\textsubscript{deviance} \mathbin{\sim} \chi^2_p$, and therefore a p-value for the significance of the model fit can be calculated. This is not reproduced, as all fits provide highly significant evidence against the null hypothesis, which assumes the fitted model to be no better than the null model.

\renewcommand{\arraystretch}{1.5}
\setlength{\tabcolsep}{0.25em}
\begin{table}
  \centering
  \caption[\Gls{glm} model summaries based on principal components for several well pairings.]{Summaries of several \gls{glm} models obtained by pairing wells corresponding to the genes MTOR (H6), PIK3R3 (K8), RIPK4 (G17) and TGFBR1 (M4) with all available scrambled wells on the same plate (J110-2D). Comparisons among scrambled wells serve as baseline (the scrambled row corresponds to well G1). Model fit is summarized by \gls{aic}, the difference in deviance and degrees of freedom (between null and fitted models), as well as out-of-sample prediction scores. The following models suffer from separated data: MTOR (A24, G1, J2), PIK3R3 (E24, G1, G23, H2, L23), RIPK4 (E2, G23, H2, J2, L1) and Scrambled (A24, G23, H24, J24, L23).}
  \label{tab:glm-1}
  \footnotesize
  \begin{tabular}{llcccccccccccc}

 &  & A2 & A24 & E2 & E24 & G1 & G23 & H2 & H24 & J2 & J24 & L1 & \multicolumn{1}{c}{L23} \\ 
\hline
\nopagebreak MTOR & \nopagebreak AIC  & \multicolumn{1}{r}{1482} & \multicolumn{1}{r}{3258} & \multicolumn{1}{r}{1639} & \multicolumn{1}{r}{3210} & \multicolumn{1}{r}{1386} & \multicolumn{1}{r}{3546} & \multicolumn{1}{r}{1590} & \multicolumn{1}{r}{3120} & \multicolumn{1}{r}{1480} & \multicolumn{1}{r}{3243} & \multicolumn{1}{r}{1922} & \multicolumn{1}{r}{3762} \\
 & \nopagebreak \textDelta\textsubscript{deviance}  & \multicolumn{1}{r}{3353} & \multicolumn{1}{r}{1726} & \multicolumn{1}{r}{3167} & \multicolumn{1}{r}{2178} & \multicolumn{1}{r}{3572} & \multicolumn{1}{r}{1427} & \multicolumn{1}{r}{2773} & \multicolumn{1}{r}{1776} & \multicolumn{1}{r}{3439} & \multicolumn{1}{r}{1531} & \multicolumn{1}{r}{2927} & \multicolumn{1}{r}{1396} \\
 & \nopagebreak \textDelta\textsubscript{df}  & \multicolumn{1}{r}{46} & \multicolumn{1}{r}{48} & \multicolumn{1}{r}{48} & \multicolumn{1}{r}{47} & \multicolumn{1}{r}{46} & \multicolumn{1}{r}{48} & \multicolumn{1}{r}{48} & \multicolumn{1}{r}{48} & \multicolumn{1}{r}{48} & \multicolumn{1}{r}{48} & \multicolumn{1}{r}{47} & \multicolumn{1}{r}{48} \\
 & \rule{0pt}{1.7\normalbaselineskip}Acc  & \multicolumn{1}{r}{0.92} & \multicolumn{1}{r}{0.77} & \multicolumn{1}{r}{0.9} & \multicolumn{1}{r}{0.81} & \multicolumn{1}{r}{0.91} & \multicolumn{1}{r}{0.75} & \multicolumn{1}{r}{0.9} & \multicolumn{1}{r}{0.79} & \multicolumn{1}{r}{0.92} & \multicolumn{1}{r}{0.79} & \multicolumn{1}{r}{0.89} & \multicolumn{1}{r}{0.76} \\
 & \nopagebreak Mcc  & \multicolumn{1}{r}{0.84} & \multicolumn{1}{r}{0.54} & \multicolumn{1}{r}{0.79} & \multicolumn{1}{r}{0.61} & \multicolumn{1}{r}{0.82} & \multicolumn{1}{r}{0.49} & \multicolumn{1}{r}{0.79} & \multicolumn{1}{r}{0.58} & \multicolumn{1}{r}{0.84} & \multicolumn{1}{r}{0.57} & \multicolumn{1}{r}{0.78} & \multicolumn{1}{r}{0.51} \\
\rule{0pt}{1.7\normalbaselineskip}PIK3R3 & \nopagebreak AIC  & \multicolumn{1}{r}{4513} & \multicolumn{1}{r}{4425} & \multicolumn{1}{r}{4027} & \multicolumn{1}{r}{4961} & \multicolumn{1}{r}{3573} & \multicolumn{1}{r}{3646} & \multicolumn{1}{r}{3452} & \multicolumn{1}{r}{4377} & \multicolumn{1}{r}{3734} & \multicolumn{1}{r}{3201} & \multicolumn{1}{r}{4198} & \multicolumn{1}{r}{3258} \\
 & \nopagebreak \textDelta\textsubscript{deviance}  & \multicolumn{1}{r}{840} & \multicolumn{1}{r}{1100} & \multicolumn{1}{r}{1292} & \multicolumn{1}{r}{1037} & \multicolumn{1}{r}{1926} & \multicolumn{1}{r}{1865} & \multicolumn{1}{r}{1353} & \multicolumn{1}{r}{1044} & \multicolumn{1}{r}{1717} & \multicolumn{1}{r}{2080} & \multicolumn{1}{r}{1171} & \multicolumn{1}{r}{2468} \\
 & \nopagebreak \textDelta\textsubscript{df}  & \multicolumn{1}{r}{48} & \multicolumn{1}{r}{49} & \multicolumn{1}{r}{50} & \multicolumn{1}{r}{49} & \multicolumn{1}{r}{49} & \multicolumn{1}{r}{49} & \multicolumn{1}{r}{49} & \multicolumn{1}{r}{49} & \multicolumn{1}{r}{50} & \multicolumn{1}{r}{49} & \multicolumn{1}{r}{49} & \multicolumn{1}{r}{49} \\
 & \rule{0pt}{1.7\normalbaselineskip}Acc  & \multicolumn{1}{r}{0.71} & \multicolumn{1}{r}{0.74} & \multicolumn{1}{r}{0.72} & \multicolumn{1}{r}{0.71} & \multicolumn{1}{r}{0.79} & \multicolumn{1}{r}{0.77} & \multicolumn{1}{r}{0.75} & \multicolumn{1}{r}{0.72} & \multicolumn{1}{r}{0.76} & \multicolumn{1}{r}{0.81} & \multicolumn{1}{r}{0.73} & \multicolumn{1}{r}{0.84} \\
 & \nopagebreak Mcc  & \multicolumn{1}{r}{0.43} & \multicolumn{1}{r}{0.48} & \multicolumn{1}{r}{0.44} & \multicolumn{1}{r}{0.4} & \multicolumn{1}{r}{0.57} & \multicolumn{1}{r}{0.54} & \multicolumn{1}{r}{0.49} & \multicolumn{1}{r}{0.44} & \multicolumn{1}{r}{0.52} & \multicolumn{1}{r}{0.63} & \multicolumn{1}{r}{0.46} & \multicolumn{1}{r}{0.68} \\
\rule{0pt}{1.7\normalbaselineskip}RIPK4 & \nopagebreak AIC  & \multicolumn{1}{r}{4012} & \multicolumn{1}{r}{5498} & \multicolumn{1}{r}{3548} & \multicolumn{1}{r}{6013} & \multicolumn{1}{r}{3117} & \multicolumn{1}{r}{4830} & \multicolumn{1}{r}{3134} & \multicolumn{1}{r}{5045} & \multicolumn{1}{r}{3321} & \multicolumn{1}{r}{4503} & \multicolumn{1}{r}{4028} & \multicolumn{1}{r}{4728} \\
 & \nopagebreak \textDelta\textsubscript{deviance}  & \multicolumn{1}{r}{1803} & \multicolumn{1}{r}{510} & \multicolumn{1}{r}{2227} & \multicolumn{1}{r}{532} & \multicolumn{1}{r}{2860} & \multicolumn{1}{r}{1163} & \multicolumn{1}{r}{2064} & \multicolumn{1}{r}{846} & \multicolumn{1}{r}{2604} & \multicolumn{1}{r}{1230} & \multicolumn{1}{r}{1806} & \multicolumn{1}{r}{1508} \\
 & \nopagebreak \textDelta\textsubscript{df}  & \multicolumn{1}{r}{48} & \multicolumn{1}{r}{49} & \multicolumn{1}{r}{49} & \multicolumn{1}{r}{49} & \multicolumn{1}{r}{48} & \multicolumn{1}{r}{49} & \multicolumn{1}{r}{49} & \multicolumn{1}{r}{49} & \multicolumn{1}{r}{50} & \multicolumn{1}{r}{49} & \multicolumn{1}{r}{49} & \multicolumn{1}{r}{49} \\
 & \rule{0pt}{1.7\normalbaselineskip}Acc  & \multicolumn{1}{r}{0.77} & \multicolumn{1}{r}{0.63} & \multicolumn{1}{r}{0.81} & \multicolumn{1}{r}{0.59} & \multicolumn{1}{r}{0.85} & \multicolumn{1}{r}{0.7} & \multicolumn{1}{r}{0.82} & \multicolumn{1}{r}{0.65} & \multicolumn{1}{r}{0.84} & \multicolumn{1}{r}{0.72} & \multicolumn{1}{r}{0.77} & \multicolumn{1}{r}{0.74} \\
 & \nopagebreak Mcc  & \multicolumn{1}{r}{0.54} & \multicolumn{1}{r}{0.25} & \multicolumn{1}{r}{0.61} & \multicolumn{1}{r}{0.18} & \multicolumn{1}{r}{0.7} & \multicolumn{1}{r}{0.41} & \multicolumn{1}{r}{0.63} & \multicolumn{1}{r}{0.28} & \multicolumn{1}{r}{0.68} & \multicolumn{1}{r}{0.42} & \multicolumn{1}{r}{0.52} & \multicolumn{1}{r}{0.48} \\
\rule{0pt}{1.7\normalbaselineskip}Scrambled & \nopagebreak AIC  & \multicolumn{1}{r}{4617} & \multicolumn{1}{r}{2405} & \multicolumn{1}{r}{4396} & \multicolumn{1}{r}{2753} & \multicolumn{1}{c}{--} & \multicolumn{1}{r}{2186} & \multicolumn{1}{r}{3707} & \multicolumn{1}{r}{2866} & \multicolumn{1}{r}{4559} & \multicolumn{1}{r}{1725} & \multicolumn{1}{r}{4702} & \multicolumn{1}{r}{1938} \\
 & \nopagebreak \textDelta\textsubscript{deviance}  & \multicolumn{1}{r}{726} & \multicolumn{1}{r}{3106} & \multicolumn{1}{r}{911} & \multicolumn{1}{r}{3230} & \multicolumn{1}{c}{--} & \multicolumn{1}{r}{3312} & \multicolumn{1}{r}{1089} & \multicolumn{1}{r}{2544} & \multicolumn{1}{r}{880} & \multicolumn{1}{r}{3541} & \multicolumn{1}{r}{655} & \multicolumn{1}{r}{3775} \\
 & \nopagebreak \textDelta\textsubscript{df}  & \multicolumn{1}{r}{48} & \multicolumn{1}{r}{48} & \multicolumn{1}{r}{49} & \multicolumn{1}{r}{48} & \multicolumn{1}{c}{--} & \multicolumn{1}{r}{48} & \multicolumn{1}{r}{49} & \multicolumn{1}{r}{49} & \multicolumn{1}{r}{49} & \multicolumn{1}{r}{47} & \multicolumn{1}{r}{48} & \multicolumn{1}{r}{48} \\
 & \rule{0pt}{1.7\normalbaselineskip}Acc  & \multicolumn{1}{r}{0.66} & \multicolumn{1}{r}{0.88} & \multicolumn{1}{r}{0.7} & \multicolumn{1}{r}{0.84} & \multicolumn{1}{c}{--} & \multicolumn{1}{r}{0.9} & \multicolumn{1}{r}{0.73} & \multicolumn{1}{r}{0.87} & \multicolumn{1}{r}{0.66} & \multicolumn{1}{r}{0.91} & \multicolumn{1}{r}{0.67} & \multicolumn{1}{r}{0.92} \\
 & \nopagebreak Mcc  & \multicolumn{1}{r}{0.32} & \multicolumn{1}{r}{0.77} & \multicolumn{1}{r}{0.41} & \multicolumn{1}{r}{0.68} & \multicolumn{1}{c}{--} & \multicolumn{1}{r}{0.8} & \multicolumn{1}{r}{0.46} & \multicolumn{1}{r}{0.73} & \multicolumn{1}{r}{0.32} & \multicolumn{1}{r}{0.83} & \multicolumn{1}{r}{0.33} & \multicolumn{1}{r}{0.84} \\
\rule{0pt}{1.7\normalbaselineskip}TGFBR1 & \nopagebreak AIC  & \multicolumn{1}{r}{3842} & \multicolumn{1}{r}{4935} & \multicolumn{1}{r}{3796} & \multicolumn{1}{r}{5522} & \multicolumn{1}{r}{3745} & \multicolumn{1}{r}{4850} & \multicolumn{1}{r}{3261} & \multicolumn{1}{r}{4998} & \multicolumn{1}{r}{3542} & \multicolumn{1}{r}{3838} & \multicolumn{1}{r}{4282} & \multicolumn{1}{r}{4676} \\
 & \nopagebreak \textDelta\textsubscript{deviance}  & \multicolumn{1}{r}{2073} & \multicolumn{1}{r}{1179} & \multicolumn{1}{r}{2080} & \multicolumn{1}{r}{1142} & \multicolumn{1}{r}{2339} & \multicolumn{1}{r}{1248} & \multicolumn{1}{r}{2024} & \multicolumn{1}{r}{996} & \multicolumn{1}{r}{2487} & \multicolumn{1}{r}{1993} & \multicolumn{1}{r}{1652} & \multicolumn{1}{r}{1672} \\
 & \nopagebreak \textDelta\textsubscript{df}  & \multicolumn{1}{r}{47} & \multicolumn{1}{r}{48} & \multicolumn{1}{r}{49} & \multicolumn{1}{r}{48} & \multicolumn{1}{r}{48} & \multicolumn{1}{r}{48} & \multicolumn{1}{r}{49} & \multicolumn{1}{r}{48} & \multicolumn{1}{r}{49} & \multicolumn{1}{r}{48} & \multicolumn{1}{r}{48} & \multicolumn{1}{r}{48} \\
 & \rule{0pt}{1.7\normalbaselineskip}Acc  & \multicolumn{1}{r}{0.79} & \multicolumn{1}{r}{0.71} & \multicolumn{1}{r}{0.79} & \multicolumn{1}{r}{0.69} & \multicolumn{1}{r}{0.81} & \multicolumn{1}{r}{0.73} & \multicolumn{1}{r}{0.83} & \multicolumn{1}{r}{0.71} & \multicolumn{1}{r}{0.83} & \multicolumn{1}{r}{0.79} & \multicolumn{1}{r}{0.76} & \multicolumn{1}{r}{0.73} \\
 & \nopagebreak Mcc  & \multicolumn{1}{r}{0.58} & \multicolumn{1}{r}{0.4} & \multicolumn{1}{r}{0.58} & \multicolumn{1}{r}{0.37} & \multicolumn{1}{r}{0.62} & \multicolumn{1}{r}{0.45} & \multicolumn{1}{r}{0.64} & \multicolumn{1}{r}{0.42} & \multicolumn{1}{r}{0.66} & \multicolumn{1}{r}{0.58} & \multicolumn{1}{r}{0.51} & \multicolumn{1}{r}{0.46} \\
\hline 
\end{tabular}


%\newcommand{\knitrDataGlm1PerSepScram}{pasteAnd(names(perfect.sep$Scrambled)[which(perfect.sep$Scrambled)])}
%\newcommand{\knitrDataGlm1PerSepMtor}{pasteAnd(names(perfect.sep$MTOR)[which(perfect.sep$MTOR)])} 
%\newcommand{\knitrDataGlm1PerSepTgfbr1}{pasteAnd(names(perfect.sep$TGFBR1)[which(perfect.sep$TGFBR1)])}
%\newcommand{\knitrDataGlm1PerSepRipk4}{pasteAnd(names(perfect.sep$RIPK4)[which(perfect.sep$RIPK4)])}
%\newcommand{\knitrDataGlm1PerSepPik3r3}{pasteAnd(names(perfect.sep$PIK3R3)[which(perfect.sep$PIK3R3)])}

\end{table}

Moving along to out-of-sample prediction scores, table \ref{tab:glm-1} shows both accuracy and \gls{mcc} values obtained by separating 20\% of data for each group from training data and evaluating predictions. Accuracy is defined as

\begin{equation}
  \text{Acc} = \frac{n_{tp} + n_{tn}}{n_p+n_n}
\end{equation}

where $n_{tp}$ represents the number of true positives, $n_{tn}$ the count of true negatives and $p$, $n$ the number of positive and negative instances, respectively. The \acrshort{mcc} \citep{Matthews1975} can be evaluated as

\begin{equation}
  \text{Mcc} = \frac{n_{tp} n_{tn} - n_{fp} n_{fn}}{\sqrt{(n_{tp} + n_{fp})(n_{tp} + n_{fn})(n_{tn} + n_{fp})(n_{tn} + n_{fn})}}
\end{equation}

and $n_{fp}$, $n_{fn}$ correspond to false positives and false negatives. Values for \gls{mcc} range from $-1$ (total disagreement) to $1$ (perfect prediction), and the midpoint $0$ indicates random prediction.

Given these model characteristics, two patterns emerge: the ability to distinguish two wells is dependent on well distance within the plate and the differences among scrambled wells, in terms of computed quality of fit estimates, are comparable to those that are observed when modeling the discrepancy between hit genes and control wells. To make the first claim, well locations are required: MTOR (H6), PIK3R3 (K8), RIPK4 (G17), TGFBR1 (M4) and Scrambled (A2, A24, E2, E24, G1, G23, H2, H24, J2, J24, L1 and L23). For both MTOR and TGFBR1, which represent early-row genes, an alternating sequence is clearly discernible and is characterized by lower AIC, larger $\textDelta\textsubscript{deviance}$ and better predictive power for wells that are closer together, while the opposite holds for comparisons among scrambled wells. The effect is less distinct for PIK3R3 and RIPK4 which both are located more towards the plate center. This is indicative of some technical artifacts contained in the data, that dominate biological features of interest. Furthermore, the excellent predictions that can be made based on membership to either one of a scrambled well pair is disconcerting, as biologically they should be equivalent. Again, technical effects dominate.

In 18 of the 59 models displayed in table \ref{tab:glm-1}, a warning regarding perfectly separated data is issued by the glm routine (in the example of MTOR, wells A24, G1 and J2). In this case, the matter is not further investigated, as it does not have obviously relevant consequences. Affected data points appear in line with problems where no complete data separation is possible and the above arguments still hold if possibly questionable data is excluded (albeit patterns are less clearly distinguishable).

Apart from the results shown, many similar investigations were performed, using other datasets and\slash or slightly different methods. Further \textit{Brucella} plates were considered, pooled \gls{sirna} experiments, libraries from Ambion and Qiagen, as were several \textit{Salmonella} plates and for some inquiries, cell population was limited to infected only. Method-wise, ridge and elastic net penalized regression (glmnet), other glm implementations, such as glm2 (\cite{Marschner2014}; uses a more robust fitting procedure), brglm (\cite{Kosmidis2013}; deals with data separation by penalized maximum likelihood) and bayesglm (\cite{Gelman2015}; regularizes coefficients though a weakly informative prior distribution), as well as step-wise model building (using the step function of the R stats package) was explored.

% bootstrapped stuff

All analysis performed clearly indicates that for the intended type of modeling, an effective normalization scheme has to be developed that is able to capture technical effects (and perhaps even spurious biological artifacts), without destroying phenotypic information coming from gene knockdown and pathogen infection. Attempts of achieving this are outlined in the following section.

\section{Data Normalization}
The large amount of technical variation, coupled with dealing with biological systems, which in turn are associated with their own inherent noise, make the analysis of high-throughput \gls{sirna} screening data a challenging endeavor, requiring thorough normalization procedures. At well level, two schemes have become standard practice, Z-scoring and B-scoring \citep{Malo2006}. The former is widely known (outside the field of \gls{sirna} analysis) and is defined as

\begin{equation}
  z_i = \frac{x_i - \mu}{\sigma}
\end{equation}

where $\mu$ represents the mean, $\sigma$ the standard deviation and $x$ the vector of features to be normalized. Applying Z-scoring both centers data around zero and scales variability to unit variance. B-scoring is more domain specific and deals with row and column effects that have been discussed previously (e.g. pipetting issues, leading to horizontal patterns or temporal effects such as decay of actin stain intensity, resulting in a vertically oriented gradient, when imaging is performed row by row). B-scoring can be expressed as

\begin{equation}
  b_{ijp} = \frac{r_{ijp}}{\mad_p} = \frac{x_{ijp} - (\Hmu_p + \widehat{r}_{ip} + \widehat{c}_{jp})}{\mad_p}
\end{equation}

and estimates for row and column effects, $\widehat{r}_{ip}$ and $\widehat{c}_{jp}$, respectively, are obtained by fitting a two-way median polish algorithm. Together with an estimate for plate average $\Hmu_p$, these three parameters are used to estimate a residual $r_{ijp}$, which divided by the \gls{mad} over the whole plate, yields the B-scored value $b_{ijp}$.

%\section{Modeling Approach}
\section{Outlook and Conclusion}