\chapter{Preface}

This thesis is submitted in partial fulfillment of the requirements for a Master's Degree in Interdisciplinary Sciences with a Major in Applied Mathematics and Computational Biology at the Swiss Federal Institute of Technology (Eidgenössische Technische Hochschule, ETH) Z\"urich. It contains work performed from April to August 2015 under the supervision of Dr.\ Anna Drewek and Prof.\ Dr.\ Peter B\"uhlmann.

All investigated data originates from the large-scale \acrfull{rnai} screening experiment \href{http://www.infectx.ch}{InfectX} (2010--2014) and the present work adds to its follow-up venture \href{http://www.targetinfectx.ch}{TargetInfectX} (ongoing). Both \acrfull{rtd} projects are funded by \href{http://www.systemsx.ch}{SystemsX}, the Swiss Initiative in Systems Biology, and set out to comprehensively study the human infectome for a set of common bacterial and viral pathogens. While InfectX was more focused on developing unified wet-lab procedures yielding comparable results over a broad range of pathogens and establishing protocols for microscopy and image analysis, TargetInfectX builds on the established datasets and places an emphasis on exploration of phenotypic space.

Picking up at the introductory quote by Benjamin Franklin, exhaustively identifying the human protein network underlying infection is an enormous task, in light of which even projects such as the InfectX program play only a fractional role towards the overall objective. It is my hope that this work may contribute an ever so small `stroke' towards felling the tree of better understanding pathogenicity in human cells.

The cover page is a graphical representation of single cell feature data as investigated. It shows the \ACRshort{mtor} well (H6) of plate J107-2C; circle size corresponds to cell area, coloring is based on mean intensity of the actin channel measured throughout entire cell bodies and filled circles are drawn for cells considered infected while outlined circles indicate healthy cells.

\section*{Declaration of Originality}

I hereby confirm that I am the sole author of the written work here enclosed and that I have compiled it in my own words. Parts excepted are corrections of form and content by the supervisors. None of this material has been submitted in any form for another degree or diploma at any institute of tertiary education other than the Swiss Federal Institute of Technology Z\"urich. The results, figures, tables and text are original, except otherwise indicated by reference to the respective authors and/or organizations.

Furthermore, I confirm that I have committed none of the forms of plagiarism as described in the ``\href{https://www.ethz.ch/content/dam/ethz/main/education/rechtliches-abschluesse/leistungskontrollen/plagiarism-citationetiquette.pdf}{Citation Etiquette}'' information sheet by ETH Z\"urich and have cited all my sources fully and verifiably in the \hyperref[ch:bibliography]{bibliography}. I have documented all methods, data and processes truthfully and I have not manipulated any data. I am aware that the work may be screened electronically for plagiarism.

\section*{Acknowledgments}

Several people were essential to the realization of this project, some of whom I wish to mention explicitly. Firstly, I would like to express my gratitude to Anna Drewek for her support and guidance throughout all stages of this project. I enjoyed the frequent meetings and benefited greatly from her insightful remarks. It was a pleasure working with her and learning from her. Along with her, I would like to thank Peter B\"uhlmann for having me be a part of his research group at the Seminar for Statistics. I am much obliged for this opportunity.

Of course, none of this would have been possible without the work put forward by the InfectX/TargetInfectX consortia. I am grateful to all who made possible this incredibly large, detailed and valuable dataset; the biologists who worked out wet-lab procedures, the machine learning experts responsible for image analysis, the modelers dealing with data normalization and all support staff, responsible for securing funding, providing computational resources and handling the administrative overhead of such a large-scale initiative. I would like to give special thanks to Mario Emmenlauer for his assistance in dealing with many issues surrounding data access and formatting.

Finally, I would like to thank both Judith Wyss an my family for all their love and support, especially during this stressful time.
