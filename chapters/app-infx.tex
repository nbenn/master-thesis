\chapter{InfectX Protocols}

\section{Materials and Methods for Wet-Lab Procedures}
\label{sec:pathogen-protocols}
The following sections describe materials and methods employed in pathogen specific protocols. This information has been published in \cite{Ramo2014} and is only reproduced for the reader's convenience.

\paragraph{\textit{B. henselae}-specific protocol.}
\textit{Bartonella henselae} ATCC49882\textsuperscript{T} \textDelta\textit{bep}G containing plasmid pCD353 \citep{Dehio1998} for IPTG-inducible expression of GFP were grown on Columbia base agar (CBA) plates supplemented with 5\% defibrinated sheep blood (Oxoid) and \SI{50}{\micro\gram\per\milli\litre} kanamycin. Bacteria were incubated at \SI{35}{\celsius} in 5\% \ce{CO2} for \SI{72}{\hour} before re-streaking them on fresh CBA and further growth for \SI{48}{\hour}. Cells were washed once after siRNA-transfection with M199 (Invitrogen)\slash 10\% \gls{fbs} using a plate washer (ELx50-16, BioTek). Cells were infected with \textit{B. henselae} at an \gls{moi} of 400 in \SI{50}{\micro\litre} of M199\slash 10\% \gls{fbs} and \SI{0.5}{\milli\Molar} IPTG (Applichem) and were incubated at \SI{35}{\celsius} in 5\% \ce{CO2} for \SI{30}{\hour}. Fixation at RT was performed using a Multidrop 384 (Thermo Scientific) to wash cells with \SI{50}{\micro\litre} of PBS, fixed in \SI{20}{\micro\litre} of 3.7\% PFA for \SI{10}{\minute}, and washed once more with \SI{50}{\micro\litre} of PBS. Staining was performed on a Biomek liquid handling platform. Fixed cells were washed twice with \SI{25}{\micro\litre} of PBS and blocked in PBS\slash 0.2\% BSA for \SI{10}{\minute}. Extracellular bacteria were labeled with a rabbit serum 2037 against B. henselae \citep{Dehio1997} and a secondary antibody goat anti rabbit A647 (Jackson Immuno) in PBS\slash 0.2\% BSA. Antibodies were incubated for \SI{30}{\minute} each and both incubations were followed by two washings with \SI{25}{\micro\litre} of PBS. Cells were then permeabilized with \SI{20}{\micro\litre} of 0.1\% Triton X-100 (Sigma) for \SI{10}{\minute} and afterwards washed twice with \SI{25}{\micro\litre} of PBS, followed by the addition of \SI{20}{\micro\litre} of staining solution (PBS containing \SI{1.5}{\micro\gram\per\milli\litre} DY-547-Phalloidin, Dyomics and \SI{1}{\micro\gram\per\milli\litre} DAPI, Roche). After \SI{30}{\minute} of incubation in the staining solution, cells were washed twice with \SI{25}{\micro\litre} PBS, followed by a final addition of \SI{50}{\micro\litre} of PBS.

\paragraph{\textit{B. abortus}-specific protocol.}
\textit{Brucella abortus} 2308 pJC43 (\textit{aphT::GFP}) \citep{Celli2005} were grown in tryptic soy broth (TSB) medium containing \SI{50}{\micro\gram\per\milli\litre} kanamycin for \SI{20}{\hour} at \SI{37}{\celsius} and shaking (\SI{100}{\rpm}) to an OD of 0.8--1.1. \SI{50}{\micro\litre} of DMEM\slash 10\% containing bacteria was added per well to obtain a final \gls{moi} of 10000 using a cell plate washer (ELx50-16, BioTek). Plates were then centrifuged at \SI{400}{\gravity} for \SI{20}{\minute} at \SI{4}{\celsius} to synchronize bacterial entry. After \SI{4}{\hour} incubation at \SI{37}{\celsius} and 5\% \ce{CO2}, extracellular bacteria were killed by exchanging the infection medium by \SI{50}{\micro\litre} medium supplemented with 10\% \gls{fbs} and \SI{100}{\micro\gram\per\milli\litre} gentamicin (Sigma). After a total infection time of \SI{44}{\hour} cells were fixed with 3.7\% PFA for \SI{20}{\minute} at RT with the cell plate washer. Staining was performed using a Biomek liquid handling platform. Cells were washed twice with PBS and permeabilized with 0.1\% Triton X (Sigma) for \SI{10}{minute}. Then, cells were washed twice with PBS, followed by addition of \SI{20}{\micro\litre} of staining solution which includes DAPI (\SI{1}{\micro\gram\per\milli\litre}, Roche) and DY-547-phalloidin (\SI{1.5}{\micro\gram\per\milli\litre}, Dyomics) in 0.5\% BSA in PBS. Cells were incubated with staining solution for \SI{30}{\minute} at RT, washed twice with PBS, followed by final addition of \SI{50}{\micro\litre} PBS.

\paragraph{\textit{L. monocytogenes}-specific protocol.}
After washing an overnight culture of \textit{L. monocytogenes} EGDe.PrfA*GFP three times with PBS, bacteria were diluted in DMEM supplemented with 1\% \gls{fbs}. Cells were infected at a \gls{moi} of 25 in \SI{30}{\micro\litre} infection medium per well. After centrifugation at \SI{1000}{\rpm} for \SI{5}{\minute} and incubation for \SI{1}{\hour} at \SI{37}{\celsius} in 5\% \ce{CO2} to allow the bacteria to enter, extracellular bacteria were killed by exchanging the infection medium by \SI{30}{\micro\litre} DMEM supplemented with 10\% \gls{fbs} and \SI{40}{\micro\gram\per\milli\litre} gentamicin (Gibco). Both medium exchange steps were carried out with a plate washer (ELx50-16, BioTek). After additional \SI{4}{\hour} at \SI{37}{\celsius} in a 5\% \ce{CO2} atmosphere, cells were fixed for \SI{15}{\minute} at RT by adding \SI{30}{\micro\litre} of 8\% PFA in PBS to each well using a multidrop 384 device (Thermo Electron Corporation). PFA was removed by four washes with \SI{500}{\micro\litre} PBS per well using the Power Washer 384 (Tecan). Fixed cells were stained for nuclei, actin and bacterially secreted InlC. First, cells were incubated for \SI{30}{\minute} with \SI{10}{\micro\litre} per well of primary staining solution (0.2\% saponin, PBS) containing rabbit derived anti-InlC serum (1:250). After four washes with \SI{40}{\micro\litre} PBS per well cells were stained with \SI{10}{\micro\litre} per well of the secondary staining solution (0.2\% saponin, PBS) containing Alexa Fluor-546 coupled anti-rabbit antibody (1:250, Invitrogen), DAPI (\SI{0.7}{\micro\gram\per\milli\litre}, Roche), and DY-647-Phalloidin (\SI{2}{\micro\gram\per\milli\litre}, Dyomics). After four washes with \SI{40}{\micro\litre} PBS per well, the cells were kept in \SI{40}{\micro\litre} PBS per well. The staining procedure was carried out with a Tecan freedom evo robot.

\paragraph{\textit{S.} typhimurium-specific protocol.}
All liquid handing stages of infection, fixation, and immunofluorescence staining were performed on a liquid handling robot (BioTek; EL406). For infection the \textit{S.} typhimurium strain S.Tm\textsuperscript{SopE\_pM975} was used. This strain is a single effector strain, only expressing SopE out of the main four SPI-1 encoded effectors (SipA, SopB, SopE2 and SopE). Additionally this strain harbors a plasmid (pM975) that expresses GFP under the control of a SPI2 (ssaG)-dependent promotor. The bacterial solution was prepared by cultivating a \SI{12}{\hour} culture in \SI{0.3}{\Molar} LB medium containing \SI{50}{\micro\gram\per\milli\litre} streptomycin and \SI{50}{\micro\gram\per\milli\litre} ampicillin. Afterwards a \SI{4}{\hour} subculture (1:20 diluted from the \SI{12}{\hour} culture) was cultivated in \SI{0.3}{\Molar} LB medium containing \SI{50}{\micro\gram\per\milli\litre} streptomycin, which reached an OD\textunderscript{\SI{600}{\nano\meter}}$\approx1.0$ after the respective \SI{4}{\hour} of incubation time. To perform the infection, \SI{16}{\micro\litre} of diluted \textit{S.} typhimurium (\gls{moi} of 80) were added to the HeLa cells. After \SI{20}{\minute} of incubation at \SI{37}{\celsius} and 5\% \ce{CO2}, the \textit{S.} typhimurium-containing media was replaced by \SI{60}{\micro\litre} DMEM\slash 10\% \gls{fbs} containing \SI{50}{\micro\gram\per\micro\litre} streptomycin and \SI{400}{\micro\gram\per\micro\litre} gentamicin to kill all remaining extracellular bacteria. After additional \SI{3}{\hour} \SI{40}{\minute} incubation at \SI{37}{\celsius} and 5\% \ce{CO2}, cells were fixed by adding \SI{35}{\micro\litre} 4\% PFA, 4\% sucrose in PBS for \SI{20}{\minute} at RT. The fixation solution was removed by adding \SI{60}{\micro\litre} PBS containing \SI{400}{\micro\gram\per\milli\litre} gentamicin. Cells were permeabilized for \SI{5}{\minute} with \SI{40}{\micro\litre} 0.1\% Triton X-100 (Sigma-Aldrich). Afterwards \SI{24}{\micro\litre} of staining solution containing DAPI (1:1000, Sigma-Aldrich) and DY-547-phalloidin (\SI{1.2}{\micro\gram\per\milli\litre}, Dyomics) was added (prepared in blocking buffer consisting of 4\% BSA and 4\% Sucrose in PBS). After \SI{1}{\hour} of incubation at RT, cells were washed three times with PBS followed by the addition of \SI{60}{\micro\litre} PBS containing \SI{400}{\micro\gram\per\milli\litre} gentamicin.

\paragraph{\textit{S. flexneri}-specific protocol.}
\textit{S. flexneri} M90T \textDelta\textit{vir}G pCK100 (PuhpT::dsRed) were harvested in exponential growth phase and coated with 0.005\% poly-L-lysine (Sigma-Aldrich). Afterwards, bacteria were washed with PBS and resuspended in assay medium (DMEM, \SI{2}{\milli\Molar} L-Glutamine, \SI{10}{\milli\Molar} HEPES). \SI{20}{\micro\litre} of bacterial suspension was added to each well with a final \gls{moi} of 15. Plates were then centrifuged for \SI{1}{\minute} at \SI{37}{\celsius} and incubated at \SI{37}{\celsius} and 5\% \ce{CO2}. After \SI{30}{\minute} of infection, \SI{75}{\micro\litre} were aspirated from each well and monensin (Sigma) and gentamicin (Gibco) were added to a final concentration of \SI{66.7}{\micro\Molar} and \SI{66.7}{\micro\gram\per\milli\litre}, respectively. After a total infection time of \SI{3.5}{\hour}, cells were fixed in 4\% PFA for \SI{10}{\minute}. Liquid handling was performed using the Multidrop 384 (Thermo Scientific) for dispension steps and a plate washer (ELx50-16, BioTek) for aspiration steps. For immunofluorescent staining, cells were washed with PBS using the Power Washer 384 (Tecan). Subsequently, cells were incubated with a mouse anti-human \gls{il-8} antibody (1:300, BD Biosciences) in staining solution (0.2\% saponin in PBS) for \SI{2}{\hour} at RT. After washing the cells with PBS, Hoechst (\SI{5}{\micro\gram\per\milli\litre}, Invitrogen), DY-495-phalloidin (\SI{1.2}{\micro\gram\per\milli\litre}, Dyomics) and Alexa Fluor 647-coupled goat anti-mouse IgG (1:400, Invitrogen) were added and incubated for \SI{1}{\hour} at RT. The staining procedure was performed using the Biomek NXP Laboratory Automation Workstation (Beckman Coulter).

\paragraph{Adenovirus-specific protocol.}
All liquid handling stages of infection, fixation, and immunofluorescence staining were performed on the automated pipetting system Well Mate (Thermo Scientific Matrix) and washer Hydrospeed (Tecan). For infection screens recombinant Ad2\_\textDelta E3B-eGFP (short Adenovirus) was utilized as described before \citep{Suomalainen2013,Yakimovich2012}. Adenovirus was added to cells at an \gls{moi} of 0.1 in \SI{10}{\micro\litre} of an infection media\slash FBS (DMEM supplemented with L-glutamine, 10\% FBS, 1\% Pen\slash Strep, Invitrogen). Screening plates were incubated at \SI{37}{\celsius} for \SI{16}{\hour}, and cells were fixed by adding \SI{21}{\micro\litre} of 16\% PFA directly to the cells in culture media for \SI{45}{\minute} at RT or long-term storage at \SI{4}{\celsius}. Cells were washed 2 times with PBS\slash \SI{25}{\milli\Molar} \ce{NH4Cl}, permeabilized with \SI{25}{\micro\litre} 0.1\% Triton X-100 (Pharmaciebiothek). After 2 washes with PBS the samples were incubated at RT for \SI{1}{\hour} with \SI{25}{\micro\litre} staining solution (PBS) containing DAPI (\SI{1}{\micro\gram\per\milli\litre}, Sigma-Aldrich) and DY-647-phalloidin (\SI{1}{\micro\gram\per\milli\litre}, Dyomics), washed 2 times with PBS and stored until imaging in \SI{50}{\micro\litre} PBS\slash \ce{NaN3}.

\paragraph{Rhinovirus-specific protocol.}
All liquid handling stages of infection, fixation, and immunofluorescence staining were performed on the automated pipetting system Well Mate (Thermo Scientific Matrix) and washer Hydrospeed (Tecan). For infection assays with human Rhinovirus serotype 1a (HRV1a) were carried out as described, except that the anti-VP2 antibody Mab 16\slash 7 was used for staining of the infected cells as described earlier \citep{Jurgeit2012,Jurgeit2010,Mosser2002}. Rhinovirus at an \gls{moi} of 8 was added to cells in \SI{20}{\micro\litre} of an infection media\slash BSA (DMEM supplemented with GlutaMAX, \SI{30}{\milli\Molar} \ce{MgCl2} and 0.2\% BSA, Invitrogen). Screening plates were incubated for \SI{7}{\hour} at \SI{37}{\celsius}, and cells were fixed by adding \SI{33}{\micro\litre} of 16\% PFA directly to the culture medium. Fixation was either for \SI{30}{\minute} at RT or long term storage at \SI{4}{\celsius}. Cells were washed twice with PBS\slash \SI{25}{\milli\Molar} \ce{H2O}, permeabilized with \SI{50}{\micro\litre} 0.2\% Triton X-100 (Sigma- Aldrich) followed by 3 PBS washes and blocking with PBS containing 1\% BSA (Fraction V, Sigma-Aldrich). Fixed and permeabilized cells were incubated at RT for \SI{1}{\hour} with diluted mabR16-7 antibody (\SI{0.45}{\micro\gram\per\milli\litre}) in PBS\slash 1\% BSA. Cells were washed 3 times with PBS and incubated with \SI{25}{\micro\litre} secondary staining solution (PBS\slash 1\% BSA) containing Alexa Fluor 488 secondary antibody (\SI{1}{\micro\gram\per\milli\litre}, Invitrogen), DAPI (\SI{1}{\micro\gram\per\milli\litre}, Sigma-Aldrich), and DY-647-phalloidin (\SI{0.2}{\micro\gram\per\milli\litre}, Dyomics). Cells were washed twice with PBS after \SI{2}{\hour} of incubation in secondary staining solution and stored in \SI{50}{\micro\litre} PBS\slash \ce{NaN3}.

\paragraph{Vacciniavirus-specific protocol.}
All liquid handing stages of infection, fixation, and immunofluorescence staining were performed on a liquid handling robot (BioTek, EL406). For infection assays a recombinant WR VACV, WR E EGFP\slash L mCherry, was utilized. For infection, media was aspirated from the RNAi-transfected cell plates and replaced with \SI{40}{\micro\litre} of virus solution per well (\gls{moi} of 0.125). Screening plates were incubated for \SI{1}{\hour} at \SI{37}{\celsius} to allow for infection, after which virus-containing media was removed and replaced with \SI{40}{\micro\litre} DMEM\slash 10\% \gls{fbs}. \SI{8}{\hour} after infection \SI{40}{\micro\litre} of DMEM\slash 10\% \gls{fbs} containing \SI{20}{\micro\Molar} cytosine arabinoside (AraC) was added to all wells to prevent virus DNA replication in secondary infected cells. \SI{24}{\hour} after infection cells were fixed by the addition of \SI{20}{\micro\litre} 18\% PFA for \SI{30}{\minute} followed by two PBS washes of \SI{80}{\micro\litre}. For immunofluorescence staining of EGFP, cells were incubated for \SI{2}{\hour} in \SI{30}{\micro\litre} primary staining solution (0.5\% Triton X-100, 0.5\% BSA, PBS) per well, containing anti-GFP antibody (1:1000). Cells were washed twice in \SI{80}{\micro\litre} PBS, followed by the addition of \SI{30}{\micro\litre} secondary staining solution (0.5\% BSA, PBS) containing Alexa Fluor 488 secondary antibody (1:1000), Hoechst (1:10000), and DY-647-phalloidin (1:1200, Dyomics). Cells were washed twice with \SI{80}{\micro\litre} PBS after \SI{1}{\hour} incubation in secondary staining solution followed by the addition of \SI{80}{\micro\litre} \ce{H2O}.

\section{Decision Trees for Infection Scoring}
\label{sec:app-dectree}
The decision trees for adenovirus and \textit{Bartonella} are shown in section~\ref{sec:infection-scoring} while the ones corresponding to the remaining pathogens (\textit{Brucella}, \textit{Listeria}, rhinovirus, \textit{Salmonella} and vaccinia virus) follow. Please refer to section~\ref{sec:infection-scoring} for more information of infection scoring, including descriptions of infection patterns upon which these decision trees are based. Table \ref{tab:dectree-thresh} shows the different thresholds that currently are in use. Due to several experimental parameters affecting intensity measurements, such as age of microscope lamp, plate position in imaging queue or quality of staining, plate-wise adjustments are necessary in some instances. Several plates can be subjected to the same values, but over all datasets, some variation exists. The values used for decision trees represent the respectively most widely used combination of parameters.

\begin{figure}[b]
\centering
\begin{tikzpicture}

\node [decision={$> 0.050$}{$\le 0.050$}] { 
  \texttt{Nuclei.Intensity\_MeanIntensity\_\\CorrPathogen}
}
[decision tree]
child { node [infected] { infected } }
child { node [decision={$> 0.055$}{$\le 0.055$}] { 
\texttt{PeriNuclei.Intensity\_\\MeanIntensity\_CorrPathogen} } 
  child { node [infected] { infected } }
  child { node [decision={$> 0.075$}{$\le 0.075$}] { 
  \texttt{Cells.Intensity\_MeanIntensity\_\\CorrPathogen} }
    child { node [infected] { infected } }
    child { node [healthy] { not infected } }
  }
};

\end{tikzpicture}
\caption[Decision tree for \textit{Brucella} infection scoring.]{Decision tree for \textit{Brucella} infection scoring. While the first two decisions are modeled to capture what is considered a normal infection pattern, the last split imposes a high threshold for cells that have failed the first two steps to still be considered infected.}
\end{figure}

\begin{figure}
\centering
\begin{tikzpicture}

\node [decision={$> 0.11$}{$\le 0.11$}] { 
  \texttt{Nuclei.Intensity\_MeanIntensity\_\\CorrInlC}
}
[decision tree]
child { node [infected] { infected } }
child { node [decision={$> 0.15$}{$\le 0.15$}] { 
\texttt{PeriNuclei.Intensity\_\\MeanIntensity\_CorrInlC} } 
  child { node [infected] { infected } }
  child { node [decision={$> 0.12$}{$\le 0.12$}] { 
  \texttt{Nuclei.Intensity\_UpperQuartile-\\Intensity\_CorrInlC} }
    child { node [infected] { infected } }
    child { node [healthy] { not infected } }
  }
};

\end{tikzpicture}
\caption[Decision tree for \textit{Listeria} infection scoring.]{The decision tree for \textit{Listeria} infection scoring is based on a channel recording InlC localization and intensity instead of targeting the bacteria themselves.}
\end{figure}

\begin{figure}
\centering
\begin{tikzpicture}

\node [decision={$> 0.085$}{$\le 0.085$}] { 
  \texttt{Nuclei.Intensity\_MeanUpperTen-\\PercentIntensity\_CorrPathogen}
}
[decision tree]
child { node [infected] { infected } }
child { node [decision={$> 0.090$}{$\le 0.090$}] { 
\texttt{PeriNuclei.Intensity\_MeanUpperTen-\\PercentIntensity\_CorrPathogen} } 
  child { node [infected] { infected } }
  child { node [decision={$> 0.125$}{$\le 0.125$}] { 
  \texttt{VoronoiCells.Intensity\_MeanUpper-\\TenPercentIntensity\_CorrPathogen} }
    child { node [infected] { infected } }
    child { node [healthy] { not infected } }
  }
};

\end{tikzpicture}
\caption[Decision tree for rhinovirus infection scoring.]{Decision tree for rhinovirus infection scoring. Using the mean of the uppermost decile of pathogen channel intensity data yields the most table results.}
\end{figure}

\begin{figure}
  \centering
  \begin{tikzpicture}
    \node [decision={$> 0.085$}{$\le 0.085$}] { 
      \texttt{Cells.Intensity\_SubCellBacteria-\\MeanIntensity\_CorrPathogen}
    }
    [decision tree]
    child { node [decision={$< 2$}{$\ge 2$}] { 
      \texttt{Cells.AreaShape\_SubCellBacteria-\\Area\_CorrPathogen} }
      child { node [infected] { infected } }
      child { node [healthy] { not infected } }
    }
    child { node [healthy] { not infected } };
  \end{tikzpicture}
  \caption[Decision tree for \textit{Salmonella} infection scoring.]{Decision tree for \textit{Salmonella} infection scoring. For a cell being considered infected, not only does the threshold for pathogen intensity throughout the cell need be exceeded but the bacteria also have to be sufficiently aggregated.}
  \label{fig:dectree-salmonella}
\end{figure}

\begin{figure}
\centering
\begin{tikzpicture}

\node [decision={$> 0.045$}{$\le 0.045$}] { 
  \texttt{Nuclei.Intensity\_MeanIntensity\_\\Corr1Pathogen}
}
[decision tree]
child { node [infected] { infected } }
child { node [decision={$> 0.050$}{$\le 0.050$}] { 
\texttt{PeriNuclei.Intensity\_\\MeanIntensity\_Corr1Pathogen} } 
  child { node [infected] { infected } }
  child { node [healthy] { not infected } }
};

\end{tikzpicture}
\caption[Decision tree for vaccinia virus infection scoring.]{Decision tree for vaccinia virus infection scoring. A separate decision tree for distinguishing primary from secondary infections has been developed but is not shown.}
\end{figure}

\renewcommand{\arraystretch}{1.1}
\setlength{\tabcolsep}{0.4em}
\begin{table}
  \caption[Different threshold values used for \gls{dtis}.]{\Gls{dtis} requires plate-wise adjustments of threshold values. Usually, a set of thresholds can be applied to multiple plates, but overall, several sets are needed, except in \textit{Bartonella} screens.}
  \label{tab:dectree-thresh}
  \footnotesize
  \vspace{3.6cm}
  % latex table generated in R 3.2.0 by xtable 1.7-4 package
% Sat Aug 22 23:14:14 2015
\begin{tabular}{lrrrrrrrrrrr}
 \rotatebox{60}{} & \rotatebox{60}{\parbox{0.17cm}{\texttt{Nuclei.Intensity\_Mean-\\Intensity\_CorrPathogen}}} & \rotatebox{60}{\parbox{0.17cm}{\texttt{PeriNuclei.Intensity\_Mean-\\Intensity\_CorrPathogen}}} & \rotatebox{60}{\parbox{0.17cm}{\texttt{Cells.Intensity\_Mean-\\Intensity\_CorrPathogen}}} & \rotatebox{60}{\parbox{0.17cm}{\texttt{Nuclei.MeanIntensity\_\\CorrInlC}}} & \rotatebox{60}{\parbox{0.17cm}{\texttt{Nuclei.UpperQuartile-\\Intensity\_CorrInlC}}} & \rotatebox{60}{\parbox{0.17cm}{\texttt{PeriNuclei.MeanIntensity\_\\CorrInlC}}} & \rotatebox{60}{\parbox{0.17cm}{\texttt{Nuclei.Intensity\_MeanUpperTen-\\PercentIntensity\_CorrPathogen}}} & \rotatebox{60}{\parbox{0.17cm}{\texttt{PeriNuclei.Intensity\_MeanUpper-\\TenPercentIntensity\_CorrPathogen}}} & \rotatebox{60}{\parbox{0.17cm}{\texttt{VoronoiCells.Intensity\_MeanUpper-\\TenPercentIntensity\_CorrPathogen}}} & \rotatebox{60}{\parbox{0.17cm}{\texttt{Cells.Intensity\_SubCellBacteria-\\MeanIntensity\_CorrPathogen}}} & \rotatebox{60}{\parbox{0.17cm}{\texttt{Cells.AreaShape\_SubCell-\\BacteriaArea\_CorrPathogen}}} \\ 
  \hline
Adeno & 0.12 & 0.13 & 0.15 &  &  &  &  &  &  &  &  \\ 
  Adeno & 0.07 & 0.08 & 0.10 &  &  &  &  &  &  &  &  \\ 
  Adeno & 0.07 & 0.09 & 0.12 &  &  &  &  &  &  &  &  \\ 
  \textit{Brucella} & 0.07 & 0.09 & 0.10 &  &  &  &  &  &  &  &  \\ 
  \textit{Brucella} & 0.04 & 0.05 & 0.07 &  &  &  &  &  &  &  &  \\ 
  \textit{Brucella} & 0.06 & 0.07 & 0.09 &  &  &  &  &  &  &  &  \\ 
  \textit{Brucella} & 0.04 & 0.04 & 0.07 &  &  &  &  &  &  &  &  \\ 
  \textit{Brucella} & 0.05 & 0.06 & 0.07 &  &  &  &  &  &  &  &  \\ 
  \textit{Listeria} &  &  &  & 0.32 & 0.34 & 0.39 &  &  &  &  &  \\ 
  \textit{Listeria} &  &  &  & 0.30 & 0.32 & 0.37 &  &  &  &  &  \\ 
  \textit{Listeria} &  &  &  & 0.22 & 0.24 & 0.29 &  &  &  &  &  \\ 
  \textit{Listeria} &  &  &  & 0.19 & 0.21 & 0.26 &  &  &  &  &  \\ 
  \textit{Listeria} &  &  &  & 0.29 & 0.31 & 0.36 &  &  &  &  &  \\ 
  \textit{Listeria} &  &  &  & 0.28 & 0.30 & 0.35 &  &  &  &  &  \\ 
  \textit{Listeria} &  &  &  & 0.21 & 0.23 & 0.28 &  &  &  &  &  \\ 
  \textit{Listeria} &  &  &  & 0.20 & 0.22 & 0.27 &  &  &  &  &  \\ 
  \textit{Listeria} &  &  &  & 0.11 & 0.12 & 0.15 &  &  &  &  &  \\ 
  Rhino &  &  &  &  &  &  & 0.08 & 0.09 & 0.12 &  &  \\ 
  Rhino &  &  &  &  &  &  & 0.12 & 0.12 & 0.17 &  &  \\ 
  Rhino &  &  &  &  &  &  & 0.09 & 0.09 & 0.12 &  &  \\ 
  Rhino &  &  &  &  &  &  & 0.09 & 0.10 & 0.13 &  &  \\ 
  Rhino &  &  &  &  &  &  & 0.10 & 0.10 & 0.14 &  &  \\ 
  \textit{Salmonella} &  &  &  &  &  &  &  &  &  & 0.10 & 2.00 \\ 
  \textit{Salmonella} &  &  &  &  &  &  &  &  &  & 0.08 & 2.00 \\ 
  \textit{Salmonella} &  &  &  &  &  &  &  &  &  & 0.09 & 2.00 \\ 
  \textit{Salmonella} &  &  &  &  &  &  &  &  &  & 0.09 & 2.00 \\ 
  \textit{Salmonella} &  &  &  &  &  &  &  &  &  & 0.10 & 2.00 \\ 
  \textit{Salmonella} &  &  &  &  &  &  &  &  &  & 0.10 & 2.00 \\ 
  Vaccinia & 0.03 & 0.04 &  &  &  &  &  &  &  &  &  \\ 
  Vaccinia & 0.04 & 0.04 &  &  &  &  &  &  &  &  &  \\ 
  Vaccinia & 0.06 & 0.07 &  &  &  &  &  &  &  &  &  \\ 
  Vaccinia & 0.03 & 0.04 &  &  &  &  &  &  &  &  &  \\ 
  Vaccinia & 0.03 & 0.03 &  &  &  &  &  &  &  &  &  \\ 
  Vaccinia & 0.03 & 0.03 &  &  &  &  &  &  &  &  &  \\ 
  Vaccinia & 0.02 & 0.03 &  &  &  &  &  &  &  &  &  \\ 
  Vaccinia & 0.04 & 0.04 &  &  &  &  &  &  &  &  &  \\ 
  Vaccinia & 0.04 & 0.05 &  &  &  &  &  &  &  &  &  \\ 
  Vaccinia & 0.03 & 0.03 &  &  &  &  &  &  &  &  &  \\ 
   \hline
\end{tabular}


\end{table}
