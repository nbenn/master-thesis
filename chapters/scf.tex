\chapter{R Package singleCellFeatures}
\label{sec:singleCellFeatures}

The format in which CellProfiler feature data is stored is only of limited suitability for exploratory data analysis. CellProfiler was originally implemented in the proprietary MATLAB language but has recently been ported to Python as version 2.x, in order to move away from the drawbacks of relying on a closed-source, commercial interpreter. Unfortunately, InfectX workflows are all based on CellProfiler 1.x and there are no current plans for updating to version 2.x. Consequently, all available single cell feature data is stored as MATLAB (Level 5) MAT-files.

The way in which storage is organized, while apt for working with a limited number of features corresponding to an entire plate, is unfitting if a large number of features belonging only to a subset of cells (e.g. all cells in a specific well) are of interest. Features are saved plate-wise in individual, gzip-compressed files, typically 1--\SI{3}{\mega\byte} in size and making the data contained in several hundred (depending on pathogen and generation the analysis pipeline, 500--700 features exist) such files available to an R session\footnote{Using R version 3.2.0 \citep{RCoreTeam2015}, installed as precompiled binaries running under Mac OS 10.10.4 on a \SI{3.4}{\giga\hertz} Intel Core i7-2600 platform with \SI{32}{\giga\byte} RAM. Whenever computational timing information is given and nothing else is specified, this is the reference system used to obtain the measurements.}, using R.matlab version 3.2.0, \cite{Bengtsson2015}, takes on the order of \SI{30}{\minute}.

As MATLAB does not constitute a tool that is particularly popular in the field of statistics and does not provide many of the convenience functions, available to R, that are much appreciated in exploratory data analysis, it was decided to convert single cell feature data as generated by CellProfiler 1.x into a format natively accessible by an R environment. Due to the amount of time involved, this cannot be performed as a first step of every analysis and owing to the amount of storage necessary, it makes little sense to be carried out beforehand for all plates. Therefore, a system is needed, capable of fetching data that is not available locally, preprocessing it for direct access by R and storing the results for future use.

Furthermore, data-structures were developed, representing the hierarchy of single cell \gls{hts} data and capable of accommodating some associated metadata. Methods for operations that are frequently performed on such data are implemented in order to simplify many analysis tasks. With growing complexity of the code-base, it was decided to create an R-package that bundles the described capabilities.

Two similar projects, cellHTS2 \citep{Boutros2006} and RNAither \citep{Rieber2009}, both hosted on Bioconductor \citep{Huber2015}, were looked at but none of them fulfilled the requirements imposed by the InfectX datasets. While cellHTS2 is designed for microarray data or \gls{sirna} data obtained by a plate reader (yielding a scalar value per well), RNAither can handle data at the single cell level. It is, however, geared towards running analysis on a single feature, obtained on a single imaging channel and cannot accommodate the heterogeneity of data available from the InfectX image analysis pipeline. In addition, RNAither is neither optimized for the large amount of data associated with several hundred features, nor does it provide the sought after tools for handling such a dataset, rather than implementing a fixed analysis procedure that can be readily applied to a single intensity feature. The newly developed singleCellFeatures therefore constitutes a further step in the evolution of R packages for \gls{sirna} data analysis, starting with cellHTS2 which is generalized in a vertical fashion by RNAither with the increase in resolution from wells to cells, which in turn is extended horizontally by singleCellFeatures to include many different features.

Much effort during development of singleCellFeatures was spent for ensuring the necessary flexibility to accommodate any possible kind of feature and for implementing some crucial sections in a way that is efficient enough for interactive usage. The former task is achieved by allowing features to consist of a single value per well, a single value per cell or a vector of values per cell and only minimally relying feature naming conventions. This is necessary owing to the large amount of variation among InfectX datasets. The latter issue is best illustrated by an introductory example.

\input{R/scf-intro/getData}
\begin{knitrout}
\definecolor{shadecolor}{rgb}{0.969, 0.969, 0.969}\color{fgcolor}\begin{figure}
\includegraphics[width=\maxwidth]{figures/R/scf-intro/plot-scf-intro_plot-1} \caption[Visualization of cell colony edge detection by 2D binning.]{Cell colony edges are detected by 2D binning of cell center locations. Dots represent cell centers within the well H6 of plate J107-2C. Each of the nine images is segmented into 12 horizontal and 12 vertical sections yielding 144 tiles (1296 bins for the entire well). The tiles are colored according to the number of non-empty neighboring bins.}\label{fig:scf-intro_plot}
\end{figure}


\end{knitrout}


As proposed by \cite{Knapp2011} and \cite{Snijder2012}, the population context of each cell may significantly influence some morphological properties, confound phenotypic information that is measured during feature extraction and therefore has to be accounted for. They propose several features that may act as proxies to characterize aspects of population context, one of which is whether a cell is located towards the border of a colony or is surrounded by cells in all directions. In order to approximate this information from location data, an image is divided into 2-dimensional bins or facets and the number of cells per bin is counted. Cells that are located adjacent to one or more bins that are empty are considered edge cells and cells surrounded by non-empty bins are center cells. Figure \ref{fig:scf-intro_plot} visualizes the concept by color-coding facets according to the number of non-empty neighbors.

\begin{rlisting}{Calculation of population context features as implemented by \citeauthor{Knapp2011}.}{In order to detect whether a cell is located towards the border of a colony or is surrounded by neighboring cells, each well is divided into 2-dimensional bins and the number of cells per bin is counted. As implemented by \citeauthor{Knapp2011}, all bins are iterated, each of the 8 possible directions is checked for an empty neighbor and the corresponding binary value is saved to the current group of cells.}{edgepos}
\begin{knitrout}\footnotesize
\definecolor{shadecolor}{rgb}{0.969, 0.969, 0.969}\color{fgcolor}\begin{kframe}
\begin{alltt}
\hlstd{edgepos} \hlkwb{<-} \hlkwa{function}\hlstd{(}\hlkwc{x}\hlstd{,} \hlkwc{y}\hlstd{,} \hlkwc{img}\hlstd{,} \hlkwc{n}\hlstd{) \{}
  \hlstd{empty} \hlkwb{<-} \hlkwd{logical}\hlstd{()}
  \hlstd{xst} \hlkwb{<-} \hlstd{img[}\hlnum{1}\hlstd{]} \hlopt{/} \hlstd{n}
  \hlstd{yst} \hlkwb{<-} \hlstd{img[}\hlnum{2}\hlstd{]} \hlopt{/} \hlstd{n}
  \hlstd{sgrid} \hlkwb{<-} \hlkwd{matrix}\hlstd{(}\hlnum{0}\hlstd{,} \hlkwc{nrow}\hlstd{=n,} \hlkwc{ncol}\hlstd{=n)}
  \hlkwa{for} \hlstd{(i} \hlkwa{in} \hlnum{1}\hlopt{:}\hlstd{n) \{}
    \hlkwa{for} \hlstd{(j} \hlkwa{in} \hlnum{1}\hlopt{:}\hlstd{n) \{}
      \hlstd{ispos} \hlkwb{<-} \hlstd{(x} \hlopt{>} \hlstd{(i} \hlopt{-} \hlnum{1}\hlstd{)} \hlopt{*} \hlstd{xst)} \hlopt{&} \hlstd{(x} \hlopt{<=} \hlstd{(i} \hlopt{*} \hlstd{xst))} \hlopt{&}
               \hlstd{(y} \hlopt{>} \hlstd{(j} \hlopt{-} \hlnum{1}\hlstd{)} \hlopt{*} \hlstd{yst)} \hlopt{&} \hlstd{(y} \hlopt{<=} \hlstd{(j} \hlopt{*} \hlstd{yst))}
      \hlstd{sgrid[i, j]} \hlkwb{<-} \hlkwd{sum}\hlstd{(ispos)}
    \hlstd{\}}
  \hlstd{\}}

  \hlkwa{for} \hlstd{(i} \hlkwa{in} \hlnum{1}\hlopt{:}\hlstd{n) \{}
    \hlkwa{for} \hlstd{(j} \hlkwa{in} \hlnum{1}\hlopt{:}\hlstd{n) \{}
      \hlstd{ispos} \hlkwb{<-} \hlstd{(x} \hlopt{>} \hlstd{(i} \hlopt{-} \hlnum{1}\hlstd{)} \hlopt{*} \hlstd{xst)} \hlopt{&} \hlstd{(x} \hlopt{<=} \hlstd{(i} \hlopt{*} \hlstd{xst))} \hlopt{&}
               \hlstd{(y} \hlopt{>} \hlstd{(j} \hlopt{-} \hlnum{1}\hlstd{)} \hlopt{*} \hlstd{yst)} \hlopt{&} \hlstd{(y} \hlopt{<=} \hlstd{(j} \hlopt{*} \hlstd{yst))}
      \hlstd{isempty} \hlkwb{<-} \hlstd{F}
      \hlkwa{if} \hlstd{((i} \hlopt{>} \hlnum{1}\hlstd{)} \hlopt{&&} \hlstd{(j} \hlopt{>} \hlnum{1}\hlstd{)} \hlopt{&&} \hlstd{(sgrid[i} \hlopt{-} \hlnum{1}\hlstd{, j} \hlopt{-} \hlnum{1}\hlstd{]} \hlopt{==} \hlnum{0}\hlstd{))}
        \hlstd{isempty} \hlkwb{<-} \hlstd{T}
      \hlkwa{else if} \hlstd{((i} \hlopt{>} \hlnum{1}\hlstd{)} \hlopt{&&} \hlstd{(sgrid[i} \hlopt{-} \hlnum{1}\hlstd{, j]} \hlopt{==} \hlnum{0}\hlstd{))}
        \hlstd{isempty} \hlkwb{<-} \hlstd{T}
      \hlkwa{else if} \hlstd{((i} \hlopt{>} \hlnum{1}\hlstd{)} \hlopt{&&} \hlstd{(j} \hlopt{<} \hlstd{n)} \hlopt{&&} \hlstd{(sgrid[i} \hlopt{-} \hlnum{1}\hlstd{, j} \hlopt{+} \hlnum{1}\hlstd{]} \hlopt{==} \hlnum{0}\hlstd{))}
        \hlstd{isempty} \hlkwb{<-} \hlstd{T}
      \hlkwa{else if} \hlstd{((j} \hlopt{>} \hlnum{1}\hlstd{)} \hlopt{&&} \hlstd{(sgrid[i, j} \hlopt{-} \hlnum{1}\hlstd{]} \hlopt{==} \hlnum{0}\hlstd{))}
        \hlstd{isempty} \hlkwb{<-} \hlstd{T}
      \hlkwa{else if} \hlstd{((j} \hlopt{<} \hlstd{n)} \hlopt{&&} \hlstd{(sgrid[i, j} \hlopt{+} \hlnum{1}\hlstd{]} \hlopt{==} \hlnum{0}\hlstd{))}
        \hlstd{isempty} \hlkwb{<-} \hlstd{T}
      \hlkwa{else if} \hlstd{((i} \hlopt{<} \hlstd{n)} \hlopt{&&} \hlstd{(j} \hlopt{>} \hlnum{1}\hlstd{)} \hlopt{&&} \hlstd{(sgrid[i} \hlopt{+} \hlnum{1}\hlstd{, j} \hlopt{-} \hlnum{1}\hlstd{]} \hlopt{==} \hlnum{0}\hlstd{))}
        \hlstd{isempty} \hlkwb{<-} \hlstd{T}
      \hlkwa{else if} \hlstd{((i} \hlopt{<} \hlstd{n)} \hlopt{&&} \hlstd{(sgrid[i} \hlopt{+} \hlnum{1}\hlstd{, j]} \hlopt{==} \hlnum{0}\hlstd{))}
        \hlstd{isempty} \hlkwb{<-} \hlstd{T}
      \hlkwa{else if} \hlstd{((i} \hlopt{<} \hlstd{n)} \hlopt{&&} \hlstd{(j} \hlopt{<} \hlstd{n)} \hlopt{&&} \hlstd{(sgrid[i} \hlopt{+} \hlnum{1}\hlstd{, j} \hlopt{+} \hlnum{1}\hlstd{]} \hlopt{==} \hlnum{0}\hlstd{))}
        \hlstd{isempty} \hlkwb{<-} \hlstd{T}
      \hlstd{empty[ispos]} \hlkwb{<-} \hlstd{isempty}
    \hlstd{\}}
  \hlstd{\}}
  \hlkwd{return}\hlstd{(empty)}
\hlstd{\}}
\end{alltt}
\end{kframe}
\end{knitrout}

\end{rlisting}

Code listing \ref{lst:edgepos} is taken from the implementation developed by \citeauthor{Knapp2011} and kindly supplied as supplement to their publication. While iterating over all facets, only exploiting vectorization within facets and heavily relying on if-else logic, might be a feasible approach, using datasets of the size the authors provide as exemplary material combined with storing the results, such an approach is impractical with datasets as produced by InfectX.

\begin{rlisting}{A more efficient implementation of calculating population context features, developed for singleCellFeatures.}{Due to the larger number of cells per well and the increase in image resolution as compared to data used in \cite{Knapp2011}, calculation of border/center population context features have to be carried out in a more efficient manner, which can be accomplished by fully vectorizing the problem.}{facetBorder}
\begin{knitrout}\footnotesize
\definecolor{shadecolor}{rgb}{0.969, 0.969, 0.969}\color{fgcolor}\begin{kframe}
\begin{alltt}
\hlstd{facetBorder} \hlkwb{<-} \hlkwa{function}\hlstd{(}\hlkwc{x}\hlstd{,} \hlkwc{y}\hlstd{,} \hlkwc{img}\hlstd{,} \hlkwc{facet}\hlstd{) \{}
  \hlstd{facet.size} \hlkwb{<-} \hlstd{img} \hlopt{/} \hlstd{facet}
  \hlcom{# calculate facets (2d binning)}
  \hlstd{x.bin} \hlkwb{<-} \hlkwd{ceiling}\hlstd{(x} \hlopt{/} \hlstd{facet.size[}\hlnum{1}\hlstd{])}
  \hlstd{y.bin} \hlkwb{<-} \hlkwd{ceiling}\hlstd{(y} \hlopt{/} \hlstd{facet.size[}\hlnum{2}\hlstd{])}
  \hlcom{# initialize empty grid/border matrices}
  \hlstd{grid} \hlkwb{<-} \hlkwd{matrix}\hlstd{(}\hlnum{0}\hlstd{, facet[}\hlnum{2}\hlstd{], facet[}\hlnum{1}\hlstd{])}
  \hlcom{# calculate col-major grid index for each object}
  \hlstd{index} \hlkwb{<-} \hlstd{y.bin} \hlopt{+} \hlstd{(x.bin} \hlopt{-} \hlnum{1}\hlstd{)} \hlopt{*} \hlstd{facet[}\hlnum{2}\hlstd{]}
  \hlcom{# summarize as counts}
  \hlstd{counts} \hlkwb{<-} \hlkwd{table}\hlstd{(index)}
  \hlcom{# fill grid with counts}
  \hlstd{grid[}\hlkwd{as.numeric}\hlstd{(}\hlkwd{names}\hlstd{(counts))]} \hlkwb{<-} \hlstd{counts}
  \hlstd{grid.res} \hlkwb{<-} \hlstd{grid}
  \hlstd{grid} \hlkwb{<-} \hlstd{grid} \hlopt{>} \hlnum{0}
  \hlcom{# extend grid with a frame of ones}
  \hlstd{grid.ext} \hlkwb{<-} \hlkwd{rbind}\hlstd{(}\hlkwd{rep}\hlstd{(}\hlnum{1}\hlstd{, (facet[}\hlnum{1}\hlstd{]} \hlopt{+} \hlnum{2}\hlstd{)),}
                    \hlkwd{cbind}\hlstd{(}\hlkwd{rep}\hlstd{(}\hlnum{1}\hlstd{, facet[}\hlnum{2}\hlstd{]), grid,} \hlkwd{rep}\hlstd{(}\hlnum{1}\hlstd{, facet[}\hlnum{2}\hlstd{])),}
                    \hlkwd{rep}\hlstd{(}\hlnum{1}\hlstd{, (facet[}\hlnum{1}\hlstd{]} \hlopt{+} \hlnum{2}\hlstd{)))}
  \hlcom{# set up stencil}
  \hlstd{row} \hlkwb{<-} \hlkwd{rep}\hlstd{(}\hlkwd{rep}\hlstd{(}\hlnum{1}\hlopt{:}\hlstd{facet[}\hlnum{2}\hlstd{], facet[}\hlnum{1}\hlstd{]))}
  \hlstd{col} \hlkwb{<-} \hlkwd{rep}\hlstd{(}\hlnum{1}\hlopt{:}\hlstd{facet[}\hlnum{1}\hlstd{],} \hlkwc{each}\hlstd{=facet[}\hlnum{2}\hlstd{])}
  \hlstd{colP1} \hlkwb{<-} \hlstd{col} \hlopt{+} \hlnum{1}
  \hlstd{colM1} \hlkwb{<-} \hlstd{col} \hlopt{-} \hlnum{1}
  \hlstd{rowP1} \hlkwb{<-} \hlstd{row} \hlopt{+} \hlnum{1}
  \hlstd{rowP2} \hlkwb{<-} \hlstd{row} \hlopt{+} \hlnum{2}
  \hlstd{nrowP} \hlkwb{<-} \hlstd{facet[}\hlnum{2}\hlstd{]} \hlopt{+} \hlnum{2}
  \hlstd{stencil} \hlkwb{<-} \hlkwd{cbind}\hlstd{(row}   \hlopt{+} \hlstd{colM1} \hlopt{*} \hlstd{nrowP,} \hlcom{# northwest neighbor}
                   \hlstd{row}   \hlopt{+} \hlstd{col}   \hlopt{*} \hlstd{nrowP,} \hlcom{# north neighbor}
                   \hlstd{row}   \hlopt{+} \hlstd{colP1} \hlopt{*} \hlstd{nrowP,} \hlcom{# northeast neighbor}
                   \hlstd{rowP1} \hlopt{+} \hlstd{colP1} \hlopt{*} \hlstd{nrowP,} \hlcom{# west neighbor}
                   \hlstd{rowP2} \hlopt{+} \hlstd{colP1} \hlopt{*} \hlstd{nrowP,} \hlcom{# east neighbor}
                   \hlstd{rowP2} \hlopt{+} \hlstd{col}   \hlopt{*} \hlstd{nrowP,} \hlcom{# southwest neighbor}
                   \hlstd{rowP2} \hlopt{+} \hlstd{colM1} \hlopt{*} \hlstd{nrowP,} \hlcom{# south neighbor}
                   \hlstd{rowP1} \hlopt{+} \hlstd{colM1} \hlopt{*} \hlstd{nrowP)} \hlcom{# southeast neighbor}
  \hlcom{# apply stencil row-wise to grid}
  \hlstd{border} \hlkwb{<-} \hlkwd{apply}\hlstd{(stencil,} \hlnum{1}\hlstd{,} \hlkwa{function}\hlstd{(}\hlkwc{ind}\hlstd{,} \hlkwc{mat}\hlstd{) \{}
    \hlkwd{return}\hlstd{(}\hlkwd{sum}\hlstd{(mat[ind]))}
  \hlstd{\},} \hlkwd{as.numeric}\hlstd{(grid.ext))}
  \hlcom{# map col-major object index to border array}
  \hlkwd{return}\hlstd{(border[index])}
\hlstd{\}}
\end{alltt}
\end{kframe}
\end{knitrout}

\end{rlisting}



\newcommand{\knitrScfBenchmarkFacetMean}{\SI{7.77}{\milli\second}}
\newcommand{\knitrScfBenchmarkFacetSd}{\SI{2.69}{\milli\second}}
\newcommand{\knitrScfBenchmarkFacetTotal}{\SI{2.98} s}
\newcommand{\knitrScfBenchmarkEdgeMean}{\SI{515.85}{\milli\second}}
\newcommand{\knitrScfBenchmarkEdgeSd}{\SI{14.66}{\milli\second}}
\newcommand{\knitrScfBenchmarkEdgeTotal}{\SI{198.08}{\second}}
\newcommand{\knitrScfBenchmarkSpeedup}{66}


Times for my version are mean \knitrScfBenchmarkFacetMean\ (with sd \knitrScfBenchmarkFacetSd) and total runtime for a plate is \knitrScfBenchmarkFacetTotal, while theirs runs with mean \knitrScfBenchmarkEdgeMean\ (sd \knitrScfBenchmarkEdgeSd) and for a plate \knitrScfBenchmarkEdgeTotal.