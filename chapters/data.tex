\chapter{Data}

\section{Infection Scoring}
\label{sec:infection-scoring}

\begin{figure}
\centering
\begin{tikzpicture}

\node [decision={$> 0.075$}{$\le 0.075$}] {
  \texttt{Cells.Intensity\_MeanIntensity\_\\CorrPathogen}
}
[decision tree]
child { node [outcome] { infected } }
child { node [decision={$> 0.085$}{$\le 0.085$}] { 
\texttt{PeriNuclei.Intensity\_MeanIntensity\_\\CorrPathogen} } 
  child { node [outcome] { infected } }
  child { node [decision={$> 0.115$}{$\le 0.115$}] { 
  \texttt{Nuclei.Intensity\_MeanIntensity\_\\CorrPathogen} } 
    child { node [outcome] { infected } }
    child { node [outcome] { not infected } }
  }
};

\end{tikzpicture}
\caption[Decision tree for adenovirus infection scoring.]{For adenovirus infection scoring, the decision tree classifier checks if enough pathogen is detected within the cell body, the perinuclear region or the nucleus. The threshold decreases as the region of interest concentrates on areas associated with progressively involved in infection.}
\end{figure}

\begin{figure}
\centering
\begin{tikzpicture}

\node [decision={$> 0.800$}{$\le 0.800$}] { 
  \texttt{Invasomes.Intensity\_MaxIntensity\_\\Corr1Pathogen}
}
[decision tree]
child { node [outcome] { infected } }
child { node [decision={$> 0.075$}{$\le 0.075$}] { 
\texttt{Invasomes.Intensity\_MeanIntensity\_\\Corr1Pathogen} } 
  child { node [outcome] { infected } }
  child { node [decision={$> 0.085$}{$\le 0.085$}] { 
  \texttt{Invasomes.Intensity\_UpperQuartile-\\Intensity\_Corr1Pathogen} }
    child { node [outcome] { infected } }
    child { node [decision={$> 12.00$}{$\le 12.00$}] { 
    \texttt{Invasomes.AreaShape\_Area} }
      child { node [decision={$< 1000$}{$\ge 1000$}] { 
        \texttt{Invasomes.AreaShape\_Area} }
        child { node [decision={$< 0.060$}{$\ge 0.060$}] { 
          \texttt{Invasomes.RadialDistribution\_FracAtD\_\\Corr1Actin\_1} }
          child { node [outcome] { infected } }
          child { node [outcome] { not infected } }
        }
        child { node [outcome] { not infected } }
      }
      child { node [outcome] { not infected } }
    }
  }
};

\end{tikzpicture}
\caption[Decision tree for \textit{Bartonella} infection scoring.]{Decision tree for \textit{Bartonella} infection scoring. In order to detect bona-fide invasomes, the first three or-linked decisions assemble a list of candidates while the following three and-linked decisions discard some erroneously included instances. In order to obtain the desired cell-based infection score, invasomes are mapped to cellular objects in a subsequent step.}
\end{figure}
