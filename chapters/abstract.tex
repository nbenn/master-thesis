\chapter{Abstract}

Infectious diseases are among the leading causes of death worldwide and the evolution of antimicrobial resistance poses a troubling development in cases where our only effective line of defense is based on distribution of antibiotic agents. One possible way out of this perilous situation comes by the alternative approach of host directed therapeutics, which in turn warrants the meticulous study of the human infectome. Therefore, large-scale studies such as genome-wide \acrshort{sirna} knockdown experiments as performed by the InfectX\slash TargetInfectX consortia are of great importance.

The richness of datasets resulting from image-based high throughput \acrshort{rnai} screens permits a broad range of possible analysis approaches to be employed. The present study investigates cellular phenotypes as induced by gene knockdown, with a focus on the effect of pathogen infection by applying \acrfullpl{glm} to single cell measurements. In order to simplify handling of such datasets, an R package is presented that is capable of fetching queried data from a centralized data store and producing a data structure capable of efficiently representing the logic of an assay plate. Convenience functions to preprocess, manipulate and normalize the resulting objects are provided, as is a caching system that helps to significantly speed up common operations.

\acrshort{glm} analysis of phenotypic response from knockdown and infection was attempted, but did not yield satisfactory results, most probably due to issues with data normalization. In order to facilitate the simultaneous study of measurements originating from multiple assay plates, several normalization schemes were explored, including Z- and B-scoring, as well as modeling technical artifacts with \acrfull{mars}. While some improvements of data quality were observed, experimental sources of error could not be sufficiently controlled for meaningful \acrshort{glm} regression.