\chapter{SingleCellFeatures Manual}
\label{ch:scf-manual}

This appendix complements chapter \ref{ch:singlecellfeatures}, which focuses of architectural and implementation aspects of the presented R package, with some practical guidance on how to install and maintain singleCellFeatures. Furthermore, some examples of how various functionality provided by singleCellFeatures can be used in practice will be given.

\section{Package Installation}
\label{sec:install-package}
All R code is available on \href{https://github.com/nbenn/singleCellFeatures}{github} and can be directly installed from an R session through the devtools package. External requirements are \href{http://zlib.net/pigz/}{pigz}, a parallel implementation of gzip and access to an openBIS \citep{Bauch2011} instance via the corresponding Java command line tool, which has to be compiled and installed locally.

\begin{rflow}
install.packages("devtools")
library(devtools)

install_github("nbenn/singleCellFeatures")
\end{rflow}

Alternatively the package can be \href{https://github.com/nbenn/singleCellFeatures/archive/master.zip}{downloaded} and installed manually by running the following commands in a shell (dependent on where the zip file was downloaded to).

\begin{bashflow}
unzip ~/Downloads/singleCellFeatures-master.zip
R CMD INSTALL --no-multiarch --with-keep.source \
  ~/Downloads/singleCellFeatures-master
\end{bashflow}

Some setup dependent information has to be provided, all of which is stored in a yaml formatted configuration file. The default location of this config file is \mintinline{text}{~/.singleCellFeaturesConfig}. This can be changed on a per-session basis using the function \mintinline{text}{configPathSet()} of more permanently, using an \mintinline{text}{.Rprofile} file. In addition, singleCellFeatures provides a function to generate a template file that can be edited to suit the current setup.

\begin{rflow}
## if no config file is present, set one up
# set the config file location
configPathSet("path/to/where/you/want/your/config.yaml")
# create a template file
configInit()
# using a text editor, modify this file for your system

## for inter-session persistence, add the following to your .Rprofile
options(singleCellFeatures.configPath = "path/to/your/config.yaml")
\end{rflow}



\begin{rlisting}{float=t}{Structure of the configuration file for singleCellData.}{In order for singleCellData to be correctly configured for a given system, several settings can be adjusted through a yaml-based configuration file.}{config}
  \inputminted[fontsize=\footnotesize,linenos,numbersep=4pt,style=knitr]{yaml}{data/config.yaml}
\end{rlisting}

The configuration file structure is shown in listing \ref{lst:config}. The two entries under \mintinline{yaml}{dataStorage} specify local file-system paths to be used for storing downloaded data and metadata. The first location (\mintinline{yaml}{dataDir}) should be chosen such that it points to a location on a volume with several \si{\giga\byte}s of free storage, in order to be able to hold a couple of plates. A complete plate requires 1--\SI{2}{\giga\byte} of storage and having upwards of \SI{50}{\giga\byte} available is recommended. The second path (\mintinline{yaml}{metaDir}) specifies the location of metadata files, the generation of which is described in section \ref{sec:update-metadata}. The section \mintinline{yaml}{beeDownloader} is concerned with the location of the Java command line tool for accessing openBIS data, which in case of InfectX is called \href{https://wiki.systemsx.ch/pages/viewpage.action?title=InfectX+Single+Cell+Data+Access&spaceKey=InfectXRTD}{BeeDataSetDownloader}. Both the executable and a folder containing several JAR-files supplied with openBIS are required. Login credentials for openBIS access can be specified in the following section (\mintinline{yaml}{openBIS}) and the final keyword group holds the path to the local source of this package. It is only used to update the databases in \mintinline{text}{/data} (see section \ref{sec:update-metadata}) and therefore will not be needed in production environments, unless the metadata that comes with singleCellData is outdated or this package is used for data not produced by InfectX.

%\section{Working Example}

\section{Metadata Databases}
\label{sec:update-metadata}
Metadata files are used in two separate contexts, the first being generation of lookup tables for dataset searches and feature availability checks and the second is involved with compilation of relevant metadata upon import of new data (see section \ref{sec:data-access}). Three different types of CSV-based files are required, a plate database, feature lists and aggregate files, creation and storage of which is described in the following paragraphs.

For production use of singleCellFeatures, only aggregate files are required, as the other two file types are available in processed form as package data files, distributed with the package. All the metadata contained in aggregate files could unfortunately not be supplied directly with the package due to large size and more importantly, because they hold information that currently may not be released to the public as a whole (e.g. sequences of all \glspl{sirna} used throughout all screens, only a subset of which is currently published).

\subsection{Plate Database}
\label{sec:plate-database}
The purpose of this file is creating a table of all plates that have associated single cell feature data available. This information is used for assessing the coverage of actual metadata and for lookup of data location both locally and on openBIS. Knowing how much of the data is represented in metadata files is important, as only that subset is available to singleCellFeatures. Upon loading the package in an R session, the extent of coverage is surveyed and reported in order to warn the user when, for example, outdated metadata is used that only contains an older set of available data.

\paragraph{Creation.}
In openBIS, choose \mintinline{text}{Browse > Data Set Search}, select the drop-down option \mintinline{text}{Data Set Type} and put in \mintinline{text}{cc} as keyword for searching for all single cell feature datasets. Make sure all available columns are displayed by selecting \mintinline{text}{Settings} and checking all \mintinline{text}{Visible?} boxes. Finally choose \mintinline{text}{Export} to save the table to disk.

\paragraph{Names and content.}
The file is expected to be named as \mintinline{text}{HCS_ANALYSIS_CELL_ FEATURES_CC_MAT.tsv} and be located directly at the path specified as \mintinline{yaml}{metaDir}. The corresponding database for the R package is located in the package \mintinline{text}{/data} directory, is saved as \mintinline{text}{/plateDatabase.rda} and the object that is attached when calling \mintinline{text}{data(plateDatabase)} in R is named \mintinline{text}{plate.database}. The table contains columns \mintinline{text}{Barcode}, \mintinline{text}{Space}, \mintinline{text}{Group}, \mintinline{text}{Experiment} and \mintinline{text}{DataID}, which for each barcode, defines the location of associated single cell feature datasets within openBIS and the local cache hierarchy, which mirrors the structure on openBIS.

\subsection{Feature Lists}
\label{sec:feature-lists}
For most of the time during development of singleCellFeatures, work on storage infrastructure at the University of Basel introduced issues when downloading datasets that could lead to features missing from the downloaded data without the user being warned about this. A simple fix is provided in the form of a feature list database that specifies a set of features that are expected to be present for each pathogen. This approach will report both false positives (features that are expected but not actually available for the given dataset) and false negatives (features that are not expected but still available), as the set of available features not only depends on pathogen, but also on the state of the analysis pipeline used for feature extraction.

A more sophisticated solution involves querying the metadata database of openBIS for all feature files per plate and the increased granularity would consequently eliminate false reports. Such an approach was contemplated but as of yet could not be implemented due to time constraints.

\paragraph{Creation.}
In openBIS, for each project (e.g. \mintinline{text}{ADENO_TEAM}), display all experiments, choose the most recent (which seems like a regular screen, e.g. \mintinline{text}{ADENO- AU-CV2}), choose any plate (all are assumed to contain the same set of features) and list all available data sets sorted by data set type. Pick the most recent data set of type \mintinline{text}{HCS_ANALYSIS_CELL_FEATURES_CC_MAT} and list all files belonging to that data set. This view is then copy-pasted into a text editor and all files that do not end in \mintinline{text}{.mat} are removed by hand (usually only 1--3 files).

\paragraph{Names and content.}
The raw data files used for updating the database associated with singleCellFeatures are expected to be located in a subdirectory to \mintinline{yaml}{metaDir} and saved as \mintinline{text}{PATHOGEN-*.txt} (for example \mintinline{text}{ADENO-AU-CV2.txt}). The pathogen name should be capitalized and specification of the screen that was used for obtaining the information may follow but is not required. Each line contains the name of a feature that will be expected to be present in all screens of the given pathogen, followed by a space, separating the name from further information which will be ignored. The resulting R data file is named \mintinline{text}{featureDatabase.rda}, the object name is \mintinline{text}{feature.database} and represents the data as a list of character vectors with slots for pathogens.

\subsection{Aggregate Files}
\label{sec:aggregate-files}
This set of files is most important, as it is required for correct operation of singleCellFeatures, whereas others are only needed for updating the corresponding data files distributed with the package itself. All metadata that is used, both for searching for datasets from within singleCellFeatures, as well as for creating \mintinline{text}{MetaData} objects, originates from aggregate files.

\paragraph{Creation.}
Many different types of aggregate files are available from the InfectX openBIS instance, compiled with different aggregation procedures and only the recent most generation contains all required columns. For accessing the files, in openBIS, choose the \mintinline{text}{Data Sets} tab under \mintinline{text}{INFECTX/_COMMOM/ REPORTS}) and sort by code. The files produced by the current aggregation procedure (imported into openBIS at the end of May 2015) are the best choice and column naming expectations of singleCellFeatures are based on this generation of aggregates (examples include \href{https://infectx.biozentrum.unibas.ch/openbis/index.html#entity=DATA_SET&permId=20150522094328451-3135287}{\textit{Brucella}} and \href{https://infectx.biozentrum.unibas.ch/openbis/index.html#entity=DATA_SET&permId=20150522100633413-3135295}{\textit{Salmonella}}).

\paragraph{Names and content.}
Aggregate files are expected to reside in a subdirectory under \mintinline{yaml}{metaDir}, called \mintinline{text}{Aggregates} and follow the naming pattern \mintinline{text}{Pathogen Report_*.csv}, where the pathogen name is specified in camel case and the specification of openBIS document ID following an underline is optional (e.g. \mintinline{text}{AdenoReport_20150522-0936.csv}). These files are large (up to \SI{200}{\mega\byte}) and constitute of rows representing wells corresponding to screens of a given patho\-gen and 55 columns, of which \mintinline{text}{Barcode}, \mintinline{text}{BATCH}, \mintinline{text}{eCount_oCells}, \mintinline{text}{Experiment}, \mintinline{text}{GENESET}, \mintinline{text}{Group}, \mintinline{text}{ID}, \mintinline{text}{ID_openBIS}, \mintinline{text}{LIBRARY}, \mintinline{text}{Name}, \mintinline{text}{PATHOGEN}, \mintinline{text}{PLATE_QUALITY_ STATUS}, \mintinline{text}{PLATE_TYPE}, \mintinline{text}{REPLICATE}, \mintinline{text}{Seed_sequence_antisense_5_3}, \mintinline{text}{Sequence_ antisense_5_3}, \mintinline{text}{Sequence_target_sense_5_3}, \mintinline{text}{Space}, \mintinline{text}{WELL_QUALITY_STATUS}, \mintinline{text}{WellColumn}, \mintinline{text}{WellRow} and \mintinline{text}{WellType} are relevant to singeCellFeatures. Please refer to section \ref{sec:metadata-objects} for more information on the individual columns.

The corresponding lookup tables store the columns \mintinline{text}{Barcode}, \mintinline{text}{WellRow}, \mintinline{text}{Well Column}, \mintinline{text}{WellType}, \mintinline{text}{ID_openBIS}, \mintinline{text}{ID} and \mintinline{text}{Name} as \mintinline{text}{wellDatabasePathogen.rda} where the pathogen name is spelled in camel case (for example \mintinline{text}{wellDatabase Adeno.rda}) and the data frame in R is called \mintinline{text}{well.database.pathogen} where the pathogen name is in lowercase letters (e.g. \mintinline{text}{well.database.adeno}). The primary purpose of well-level metadata lookup tables is fast searching, as constant loading of large pathogen-level aggregate files is time consuming. A further type of aggregate-derived files are plate metadata caches. These are saved alongside the directory structure that organizes downloaded data and their naming scheme follows \mintinline{text}{barcode_metadata.rds} (e.g. \mintinline{text}{J101-2C_metadata.rds}). Plate-level metadata caches consist of all columns available in aggregate files and the 384 rows corresponding to the given plate and again constitute an optimization mechanism for reducing the time required for serving complete \mintinline{text}{Data} objects.

\section{Dataset search}
\label{sec:dataset-search}
Two functions for searching datasets are available, \mintinline{text}{findPlates} and \mintinline{text}{findWells}, both of which return lists of respective \mintinline{text}{DataLocation} objects. The arguments common to both functions along are \mintinline{text}{pathogens}, \mintinline{text}{experiments}, \mintinline{text}{plates}, \mintinline{text}{well. types}, \mintinline{text}{contents}, \mintinline{text}{id.openBIS}, \mintinline{text}{id.infx}, \mintinline{text}{name} and \mintinline{text}{verbose}. All, except for the last parameter, which is a logical value for increased verbosity, can accept vector valued arguments that are matched against corresponding metadata columns. Please refer to table \ref{tab:dataset-search} for more details on how values are matched and which metadata columns (see tables \ref{tab:plate-metadata} and \ref{tab:well-metadata}) are involved. 

\renewcommand{\arraystretch}{1.6}
\setlength{\tabcolsep}{0.2em}
\begin{table}
  \centering
  \caption[Arguments accepted by dataset search functions.]{Two functions for identifying datasets of interest are available, \mintinline{text}{findPlates} and \mintinline{text}{findWells}. The first set of parameters is available to both functions, while the three arguments separated at the bottom can only be specified in \mintinline{text}{findWells}. The column specifying targeted metadata fields is referring to names explained in tables \ref{tab:plate-metadata} and \ref{tab:well-metadata}.}
  \label{tab:dataset-search}
  \footnotesize
  \begin{tabular}{L{0.25\linewidth}L{0.25\linewidth}L{0.45\linewidth}}
    Parameter name &
      Metadata column &
      Description \\
    \hline 
    \mintinline{text}{pathogens = NULL} &
      \mintinline{text}{experiment.group} &
      Vector of strings, case insensitive, matched by adding the suffix \mintinline{text}{_TEAM} to the input. \\
    \mintinline{text}{experiments = NULL} &
      \mintinline{text}{experiment.name} &
      Vector of strings, applied as individual case insensitive regular expressions. \\
    \mintinline{text}{plates = NULL} &
      \mintinline{text}{plate.barcode} &
      Vector of strings, case sensitive, matched exactly. \\
    \mintinline{text}{well.types = NULL} &
      \mintinline{text}{well.type} &
      Vector of strings, applied as individual case insensitive regular expressions. \\
    \mintinline{text}{contents = NULL} &
      \mintinline{text}{sirna.name}, \mintinline{text}{gene.id} or \mintinline{text}{gene.name} &
      Vector of strings, case insensitive, each string is matched against any of the three columns. \\
    \mintinline{text}{id.openBIS = NULL} &
      \mintinline{text}{sirna.name} &
      Vector of strings, applied as individual case insensitive regular expressions. \\
    \mintinline{text}{id.infx = NULL} &
      \mintinline{text}{gene.id} &
      Vector of strings, applied as individual case insensitive regular expressions. \\
    \mintinline{text}{name = NULL} &
      \mintinline{text}{gene.name} &
      Vector of strings, applied as individual case insensitive regular expressions. \\
    \mintinline{text}{verbose = FALSE} &
      -- &
      Logical value indicating whether to produce more verbose output. \\
    \hline 
    \mintinline{text}{well.rows = NULL} &
      \mintinline{text}{well.row} &
      Vector of single characters ($\in \{ A, B, \dotsc, P \} \land \allowbreak \{ a, b, \dotsc, p \}$), matched exactly. \\
    \mintinline{text}{well.cols = NULL} &
      \mintinline{text}{well.col} &
      Vector of single integers ($\in \{ 1, 2, \dotsc, 24 \}$), matched exactly. \\
    \mintinline{text}{well.names = NULL} &
      \mintinline{text}{well.row} and \mintinline{text}{well.col} &
      Vector of strings ($\in \{ A1, A2, \dotsc, A24, B1, \dotsc, \allowbreak P24 \}$), matched exactly. \\
    \hline 
  \end{tabular}
\end{table}

In order to achieve quick lookups, not the metadata files themselves are subject to search, but rather an indexed version in the form of a plate database (see section \ref{sec:plate-database}) and pathogen specific well databases (see section \ref{sec:aggregate-files}). This reduces the time frame required for performing searches considerably, especially in pathogen spanning cases, as the relevant data files that have to be loaded are 1--\SI{3}{\mega\byte}, which corresponds to a 100-fold reduction in size, and therefore can be attached instantly.