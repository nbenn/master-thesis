%% (Master) Thesis template
%% Template version used: v1.4 (CADMO, ETHZ)
%%
%% Largely adapted from Adrian Nievergelt's template for the ADPS
%% (lecture notes) project.

%%%%%%%%%%%%%%%%%%%%%%%%%%%%%%%%%%%%%%%%%%%%%%%%%%%%%%%%%%%%%%%%%%%%%%%%%%%%%%%
%%%  Document settings                                                      %%%
%%%%%%%%%%%%%%%%%%%%%%%%%%%%%%%%%%%%%%%%%%%%%%%%%%%%%%%%%%%%%%%%%%%%%%%%%%%%%%%

%% We use the memoir class because it offers a many easy to use features.
\documentclass[11pt,a4paper,titlepage]{memoir}

%% LaTeX Font encoding -- DO NOT CHANGE
\usepackage[OT1]{fontenc}

%% Babel provides support for languages.  'english' uses British
%% English hyphenation and text snippets like "Figure" and
%% "Theorem". Use the option 'ngerman' if your document is in German.
%% Use 'american' for American English.  Note that if you change this,
%% the next LaTeX run may show spurious errors.  Simply run it again.
%% If they persist, remove the .aux file and try again.
\usepackage[english]{babel}

%% Input encoding 'utf8'. In some cases you might need 'utf8x' for
%% extra symbols. Not all editors, especially on Windows, are UTF-8
%% capable, so you may want to use 'latin1' instead.
\usepackage[utf8]{inputenc}

%% This changes default fonts for both text and math mode to use Herman Zapfs
%% excellent Palatino font.  Do not change this.
\usepackage[sc]{mathpazo}

%% The AMS-LaTeX extensions for mathematical typesetting.  Do not
%% remove.
\usepackage{amsmath,amssymb,amsfonts,mathrsfs}

%% NTheorem is a reimplementation of the AMS Theorem package. This
%% will allow us to typeset theorems like examples, proofs and
%% similar.  Do not remove.
%% NOTE: Must be loaded AFTER amsmath, or the \qed placement will
%% break
\usepackage[amsmath,thmmarks]{ntheorem}

%% LaTeX' own graphics handling
\usepackage{graphicx}

%% This allows you to add .pdf files. It is used to add the
%% declaration of originality.
\usepackage{pdfpages}

%% Some more packages that you may want to use.  Have a look at the
%% file, and consult the package docs for each.
%% See the TeXed file for more explanations

%% [OPT] Multi-rowed cells in tabulars
\usepackage{multirow}

%% [REC] Intelligent cross reference package. This allows for nice
%% combined references that include the reference and a hint to where
%% to look for it.
\usepackage{varioref}

%% [OPT] Easily changeable quotes with \enquote{Text}
\usepackage[german=swiss]{csquotes}

%% [REC] Format dates and time depending on locale
\usepackage{datetime}

%% [OPT] Provides a \cancel{} command to stroke through mathematics.
%\usepackage{cancel}

%% [NEED] This allows for additional typesetting tools in mathmode.
%% See its excellent documentation.
\usepackage{mathtools}

%% [ADV] Conditional commands
%\usepackage{ifthen}

%% [OPT] Manual large braces or other delimiters.
%\usepackage{bigdelim, bigstrut}

%% [REC] Alternate vector arrows. Use the command \vv{} to get scaled
%% vector arrows.
\usepackage[h]{esvect}

%% [NEED] Some extensions to tabulars and array environments.
\usepackage{array}

%% [OPT] Postscript support via pstricks graphics package. Very
%% diverse applications.
%\usepackage{pstricks,pst-all}

%% [?] This seems to allow us to define some additional counters.
%\usepackage{etex}

%% [ADV] XY-Pic to typeset some matrix-style graphics
%\usepackage[all]{xy}

%% [OPT] This is needed to generate an index at the end of the
%% document.
%\usepackage{makeidx}

%% [OPT] Fancy package for source code listings.  The template text
%% needs it for some LaTeX snippets; remove/adapt the \lstset when you
%% remove the template content.
\usepackage[outputdir=.build]{minted}
\usepackage[most]{tcolorbox}
\usepackage{float}
\usepackage{afterpage}

%% [REC] Fancy character protrusion.  Must be loaded after all fonts.
\usepackage[activate]{pdfcprot}

%% [REC] Nicer tables.  Read the excellent documentation.
\usepackage{booktabs}

%% bibliography related
\usepackage[natbib,
  authordate,
  backend=biber,
  isbn=false,
  numbermonth=false]{biblatex-chicago}
\addbibresource{mendeley.bib}
%% non-utf8 characters imported by mendeley into the bibliography
\DeclareUnicodeCharacter{2010}{-}
\DeclareUnicodeCharacter{0301}{/}
%% hide language field/url in certain types
\DeclareSourcemap{
  \maps{
    \map{
      \step[fieldset=language, null]
    }
    \map{
      \pertype{article}
      \pertype{book}
      \pertype{inbook}
      \step[fieldset=url, null]
    }
  }
}
% enable linebreaks after the doi keyword
\DeclareFieldFormat{doi}{%
  \textrm{doi}\addcolon\allowbreak
  \ifhyperref
    {\href{http://dx.doi.org/#1}{\nolinkurl{#1}}}
    {\nolinkurl{#1}}}
% settings for linebreak penalties in urls/dois
\setcounter{biburllcpenalty}{100}
\setcounter{biburlucpenalty}{200}
% remove dashes in repeat authors
\makeatletter
\AtEveryBibitem{%
  \global\undef\bbx@lasthash%
  \clearfield{extrayear}}
\makeatother
%% use same for for urls/dois
\urlstyle{same}

%% glossary/acronyms
\usepackage[nogroupskip, toc]{glossaries}
\usepackage{glossary-mcols}
\glsdisablehyper
\makeglossaries
\glsenableentrycount
\renewcommand*{\glspostdescription}{} % Removes dots at the end of each entry.

%% greek letters in text mode
\usepackage[euler]{textgreek}

%% checkmark and other symbols
%\usepackage{pifont}

%% rotate cells in tables
\usepackage[figuresright]{rotating}

%% for degree symbols and other unit stuffs
\usepackage[binary-units=true]{siunitx}

%% inline lists
\usepackage[inline]{enumitem}

%% stoichiometric formulae and chemical equations
\usepackage[version=4]{mhchem}

%% capability for multiple footnotes & footnotes in section headings
\usepackage[multiple, stable]{footmisc}

%% for indicator function 1
\usepackage{dsfont}
\usepackage{xparse}

%% Our layout configuration.  DO NOT CHANGE.
%% Memoir layout setup

%% NOTE: You are strongly advised not to change any of them unless you
%% know what you are doing.  These settings strongly interact in the
%% final look of the document.

%% Title page: redefine maketitle macro
\newlength{\logounitlength}
\setlength{\logounitlength}{0.08mm}

\def\thesistype#1{\def\thesistypestring{#1}}
\def\thesistypestring{}
\def\thesisperiod#1{\def\thesisperiodstring{#1}}
\def\thesisperiodstring{}
\def\thesistitle#1{\def\thesistitlestring{#1}}
\def\thesistitlestring{}
\def\thesisauthor#1{\def\thesisauthorstring{#1}}
\def\thesisauthorstring{}
\def\submissiondate#1{\def\submissiondatestring{#1}}
\def\submissiondatestring{}
\def\alternatereader#1{\def\alternatereaderstring{#1}}
\def\alternatereaderstring{}
\def\mainreader#1{\def\mainreaderstring{#1}}
\def\mainreaderstring{}

\newcommand{\ETHlogo}{
  \parbox{100mm}{
    \vbox{
      \kern2pt\setlength{\unitlength}{\logounitlength}
      \begin{picture}(330,110)(0,-5)
        \thicklines
        \multiput(0,0)(2,0){16}{\line(1,4){25}}
        \multiput(1,0)(0.5,2){12}{\line(1,0){84}}
        \multiput(42,40.4)(0.5,2){11}{\line(1,0){53}}
        \multiput(19.5,78.8)(0.5,2){12}{\line(1,0){210}}
        \multiput(116,0)(2,0){16}{\line(1,4){24}}
        \multiput(180,0)(2,0){16}{\line(1,4){25}}
        \multiput(237,0)(2,0){16}{\line(1,4){25}}
        \multiput(220,39)(0.5,2){12}{\line(1,0){30}}
        \put(262.5,100){\line(1,0){30}}
      \end{picture}
      \hfill\break\sffamily\bfseries Swiss Federal Institute of Technology 
      Zurich
    }
  }
}

\newcommand{\SfSlogo}{
  \parbox{40mm}{
    \hfill\kern2pt
    \setlength{\unitlength}{\logounitlength}\begin{picture}(330,110)(0,-5)
    \end{picture}\\
    \sffamily\bfseries
    \null\hfill Seminar for\kern3mm\break
    \null\hfill \phantom{g}Statistics\kern3mm%
  }
}

\makeatletter
\def\maketitle{
  \begingroup
  \hspace*{-3pt}\raise 20pt\hbox{\ETHlogo}\hfill
    \raise 19pt\hbox{\SfSlogo}\hspace*{-10pt}
  \linebreak
  \vspace{1pt}
  {\sffamily\bfseries\noindent{\\Departement of Mathematics}}
  \par\vspace*{5pt}
  \rule{\linewidth}{.3pt}\\[1pt]
  \vspace*{-15pt}
  \begin{center}
    \Large \thesistypestring
    \hfill
    \Large \thesisperiodstring\\[2pt] 
  \end{center}
  \rule{\linewidth}{.3pt}
  \medskip
  \vspace{70pt}
  \begin{center}
    \bfseries
    \Large \thesisauthorstring\\
    \vspace{2ex plus 1ex minus 1.5ex}
    \LARGE \thesistitlestring
  \end{center}
  \vspace{\stretch{2}}
  \vspace{\stretch{3}}

  \begin{center} \large
    \rule{.5\linewidth}{.3pt} \\[8pt]
    \begin{tabular}{ll}
      Submission Date: & \submissiondatestring
    \end{tabular}
    \\[5pt] \rule{.5\linewidth}{.3pt}
    
    \vspace{2ex plus 25ex minus 1.5ex}
    
    \begin{tabular}{ll}
      Co-Adviser: & \alternatereaderstring \\
      Adviser:    & \mainreaderstring
    \end{tabular}
  \end{center}
  \vspace*{-20pt}

  \endgroup
}
\makeatother

%% Turn extra space before chapter headings off.
\setlength{\beforechapskip}{0pt}

\nonzeroparskip
\parindent=0pt
\defaultlists

%% Chapter style redefinition
\makeatletter

\if@twoside
  \pagestyle{Ruled}
  \copypagestyle{chapter}{Ruled}
\else
  \pagestyle{ruled}
  \copypagestyle{chapter}{ruled}
\fi
\makeoddhead{chapter}{}{}{}
\makeevenhead{chapter}{}{}{}
\makeheadrule{chapter}{\textwidth}{0pt}
\copypagestyle{abstract}{empty}

\makechapterstyle{bianchimod}{%
  \chapterstyle{default}
  \renewcommand*{\chapnamefont}{\normalfont\Large\sffamily}
  \renewcommand*{\chapnumfont}{\normalfont\Large\sffamily}
  \renewcommand*{\printchaptername}{%
    \chapnamefont\centering\@chapapp}
  \renewcommand*{\printchapternum}{\chapnumfont {\thechapter}}
  \renewcommand*{\chaptitlefont}{\normalfont\huge\sffamily}
  \renewcommand*{\printchaptertitle}[1]{%
    \hrule\vskip\onelineskip \centering \chaptitlefont\textbf{\vphantom{gyM}##1}\par}
  \renewcommand*{\afterchaptertitle}{\vskip\onelineskip \hrule\vskip
    \afterchapskip}
  \renewcommand*{\printchapternonum}{%
    \vphantom{\chapnumfont {9}}\afterchapternum}}

%% Use the newly defined style
\chapterstyle{bianchimod}

\setsecheadstyle{\Large\bfseries\sffamily}
\setsubsecheadstyle{\large\bfseries\sffamily}
\setsubsubsecheadstyle{\bfseries\sffamily}
\setparaheadstyle{\normalsize\bfseries\sffamily}
\setsubparaheadstyle{\normalsize\itshape\sffamily}
\setsubparaindent{0pt}

%% Set captions to a more separated style for clearness
\captionnamefont{\sffamily\bfseries\footnotesize}
\captiontitlefont{\sffamily\footnotesize}
\setlength{\intextsep}{16pt}
\setlength{\belowcaptionskip}{1pt}

%% Set section and TOC numbering depth to subsection
\setsecnumdepth{subsection}
\settocdepth{subsection}

\checkandfixthelayout

\setlength{\droptitle}{-48pt}

\makeatother

%% This defines how theorems should look. Best leave as is.
\theoremstyle{plain}
\setlength\theorempostskipamount{0pt}


%% Theorem environments.  You will have to adapt this for a German
%% thesis.
\input{setup/theoremsetup}

%% Helpful macros.
%% Custom commands
%% ===============

%% Special characters for number sets, e.g. real or complex numbers.
\newcommand{\C}{\mathbb{C}}
\newcommand{\K}{\mathbb{K}}
\newcommand{\Nat}{\mathbb{N}}
\newcommand{\Q}{\mathbb{Q}}
\newcommand{\R}{\mathbb{R}}
\newcommand{\Z}{\mathbb{Z}}
\newcommand{\X}{\mathbb{X}}

%% Fixed/scaling delimiter examples (see mathtools documentation)
\DeclarePairedDelimiter\abs{\lvert}{\rvert}
\DeclarePairedDelimiter\norm{\lVert}{\rVert}

%% Use the alternative epsilon per default and define the old one as \oldepsilon
\let\oldepsilon\epsilon
\renewcommand{\epsilon}{\ensuremath\varepsilon}

%% Also set the alternate phi as default.
\let\oldphi\phi
\renewcommand{\phi}{\ensuremath{\varphi}}

\newcommand\given[1][]{\:#1\vert\:}
\let\existstemp\exists
\let\foralltemp\forall
\renewcommand{\exists}{\ensuremath\enskip\existstemp\:}
\renewcommand{\forall}{\ensuremath\enskip\foralltemp\:}

%% excerpts from SfS texab.sty
%% ===========================
%% R program
\newcommand*{\Rp}{\textsf{R}$\;$}
%% the epsfCfile function used in an example
\makeatletter
\newcommand{\@epsFFile}[2]{\includegraphics[#1]{#2}}
\newcommand{\epsFracFile}[2]{\@epsFFile{width=#1\textwidth}{#2}}%
\newcommand{\epsFracFileRot}[3][-90]{\@epsFFile{angle=#1,width=#2\textwidth}{#3}}
\newcommand{\epsfCfile}[2]{\centerline{\epsFracFile{#1}{#2}}}%
\newcommand{\epsfCfileRot}[3][-90]{\centerline{\epsFracFileRot[#1]{#2}{#3}}}%
\makeatother

\DeclareMathOperator{\logit}{logit}
\DeclareMathOperator{\med}{med}
\DeclareMathOperator{\median}{median}
\DeclareMathOperator{\Med}{Med}
\DeclareMathOperator{\Erw}{\mathbf{E}}%-- see also \E (below)
\DeclareMathOperator{\var}{var}
\DeclareMathOperator{\Var}{Var}
\DeclareMathOperator{\Cov}{Cov}
\DeclareMathOperator{\cov}{cov}
\DeclareMathOperator{\Cor}{Corr}
\DeclareMathOperator{\cor}{corr}
\DeclareMathOperator{\se}{se}
\DeclareMathOperator{\sd}{sd}
\DeclareMathOperator{\sign}{sign}
\DeclareMathOperator{\trace}{tr}
\DeclareMathOperator{\const}{const}
\DeclareMathOperator{\diag}{diag}
%% Nachteil von diesen (wegen \limits !?): \argmax_\beta setzt \beta
%% *unterhalb*, auch inline, was  \max_\beta nicht tut
%% \newcommand{\argmin}{\mathop{\arg \min}\limits}
%% \newcommand{\argmax}{\mathop{\arg \max}\limits}
%% \newcommand{\ave}{\mathop{ave}\limits}
%% This is following http://en.wikipedia.org/wiki/Arg_max 's recommendation:
\DeclareMathOperator*{\ave}{ave}
\DeclareMathOperator*{\argmin}{arg\, min}
\DeclareMathOperator*{\argmax}{arg\, max}
%
\DeclareMathOperator{\IF}{IF}
\DeclareMathOperator{\und}{ und }
\DeclareMathOperator{\oder}{ oder }
%\def\mit{\mathrm{ mit }}  %-- causes some problems with \mbox or \boldmath !!!
\DeclareMathOperator{\for}{ for }
\DeclareMathOperator{\where}{ where }
\DeclareMathOperator{\with}{ with }
%%-------  NB.  << NEVER >> REdefine  \or !!! ----
%% Not ``\and'': this is used in \author

% - Math. Operatoren
\newcommand{\op}[1]{\mathop{#1}}%was\newcommand{\op}[1]{\kern-.2em #1\kern-.2em}
% \newcommand{\oneover}[1]{{1\over#1}\,\,}
\newcommand{\oneover}[1]{{\textstyle 1\over#1}}
\newcommand{\inv}{^{-1}}
\newcommand{\sups}[1]{^{(#1)}}
%%-- shorter, for convergence ...==== unnecessary ===
%%-- There is already \to  (in standard TeX & LaTeX !) :
\newcommand{\go}{\rightarrow%
        \typeout{use standard \TeX \verb|\to| instead of \verb|\go|}}
%% convergence in Probability, Distribution: \tendsto{P}, \tendsto{D} :
\newcommand{\tendsto}[1]{\buildrel {#1} \over \longrightarrow}
\newcommand{\wt}{\widetilde}
\newcommand{\wh}[1]{\if \sigma{#1}\widehat{\sigma}\else
{}\kern0.1em\widehat{\kern-0.1em#1}{}\fi}
\newcommand{\wb}[1]{{}\if Y#1\overline{Y}\else
\kern0.2em\overline{\kern-0.2em#1}\fi}
\newcommand{\Ssum}{\sum\nolimits}
\newcommand{\Sn}{\Ssum_{i=1}^n}
\newcommand{\Seq}[4]{#1_{#2},\,#1_{#3},\ldots,#1_{#4}}% siehe auch \Nl , \No (unten)
\newcommand{\Half}{{\textstyle 1\over2}}% siehe auch \half (unten)
\newcommand{\ul}[1]{\textbf{#1}}

\newcommand{\Hmu}{\widehat{\mu}}
\newcommand{\Halpha}{\widehat{\alpha}}
\newcommand{\Hbeta}{\widehat{\beta}}
%
\newcommand{\iid}{\mbox{ i.i.d. }}
\newcommand{\fs}{\mbox{f.s.}}
\newcommand{\Loglik}{\ell\kern-1pt\ell}

%%%------------ script und andere Spezialschriften ----------------------


\DeclareMathOperator{\N}{\mathcal{N}}% Normalverteilung
\renewcommand{\P}{\mathcal{P}}%--- RE-defining the standard \P (Paragraph)!!--
%%%
\DeclareMathOperator{\M}{\mathcal{M}}% Multinom ?
\DeclareMathOperator{\B}{\mathcal{B}}% Binom
\DeclareMathOperator{\Bern}{\mathcal{B}ernoulli}
\DeclareMathOperator{\E}{\mathcal{E}}%-- see also \Erw (above)
\DeclareMathOperator{\F}{\mathcal{F}}
\DeclareMathOperator{\Wish}{\mathcal{W}}
\newcommand{\phibf}{\boldsymbol{\phi}}%{\phi\hskip-5.4pt\phi} % bold \phi


%% Make document internal hyperlinks wherever possible. (TOC, references)
%% This MUST be loaded after varioref, which is loaded in 'extrapackages'
%% above.  We just load it last to be safe.
\usepackage[
  linkcolor=black,
  colorlinks=true,
  citecolor=black,
  filecolor=black
]{hyperref}

%% Document information
\thesistype{Master Thesis}
\thesisperiod{Spring 2015}
\thesisauthor{Nicolas Bennett}
\thesistitle{
  The title of my thesis\\
  which should be split on\\
  several lines if it is too long
}
\submissiondate{August 25th 2015}
\mainreader{Prof.\ Dr.\ Peter Bühlmann}
\alternatereader{Anna Drewek}

%%%%%%%%%%%%%%%%%%%%%%%%%%%%%%%%%%%%%%%%%%%%%%%%%%%%%%%%%%%%%%%%%%%%%%%%%%%%%%%
%%%  Document body                                                          %%%
%%%%%%%%%%%%%%%%%%%%%%%%%%%%%%%%%%%%%%%%%%%%%%%%%%%%%%%%%%%%%%%%%%%%%%%%%%%%%%%

\begin{document}

\frontmatter

%% Title page
\begin{titlingpage}
  \vspace*{-3.5cm}
  \calccentering{\unitlength}
  \begin{adjustwidth*}{\unitlength-2cm}{-\unitlength-2cm}
    \maketitle
  \end{adjustwidth*}
\end{titlingpage}

%% The abstract of your thesis.  Edit the file as needed.
\chapter{Abstract}

Infectious diseases are among the leading causes of death worldwide and the evolution of antimicrobial resistance poses a troubling development in cases where our only effective line of defense is based on distribution of antibiotic agents. One possible way out of this perilous situation comes by the alternative approach of host directed therapeutics, which in turn warrants the meticulous study of the human infectome. Therefore, large-scale studies such as genome-wide \acrshort{sirna} knockdown experiments as performed by the InfectX\slash TargetInfectX consortia are of great importance.

The richness of datasets resulting from image-based high throughput \acrshort{rnai} screens permits a broad range of possible analysis approaches to be employed. The present study investigates cellular phenotypes as induced by gene knockdown, with a focus on the effect of pathogen infection by applying \acrfullpl{glm} to single cell measurements. In order to simplify handling of such datasets, an R package is presented that is capable of fetching queried data from a centralized data store and producing a data structure capable of efficiently representing the logic of an assay plate. Convenience functions to preprocess, manipulate and normalize the resulting objects are provided, as is a caching system that helps to significantly speed up common operations.

\acrshort{glm} analysis of phenotypic response from knockdown and infection was attempted, but did not yield satisfactory results, most probably due to issues with data normalization. In order to facilitate the simultaneous study of measurements originating from multiple assay plates, several normalization schemes were explored, including Z- and B-scoring, as well as modeling technical artifacts with \acrfull{mars}. While some improvements of data quality were observed, experimental sources of error could not be sufficiently controlled for meaningful \acrshort{glm} regression.

%% TOC with the proper setup, do not change.
\cleartorecto
\tableofcontents
\mainmatter

%% Your real content!
\chapter{Introduction}

\input{R/ggplotTheme}

Infectious diseases have played an undeniably important role in human history. With human populations becoming sufficiently aggregated to sustain direct life cycle viral and bacterial infectants around 2000 BC, devastating invasions of a growing number of pathogens started to occur \citep{Dobson1996}.

One of the earliest well documented incidence of a large-scale epidemic is known as the Plague of Athens. Starting in 430 BC and lasting roughly three years, a highly infectious disease killed 75'000 to 100'000 people or 25\% of Athen's population. This catastrophic event is attributed either to smallpox, a viral infection with \textit{Variola major} of typhus, caused by \textit{Rickettsia} bacteria \citep{Littman2009}.

The bacterium \textit{Yersinia pestis} caused three major plague pandemics in the early and late middle ages, as well as in the late 19th century. Originating in northern Africa in 523 AD and spreading around the Mediterranean basin throughout the years 541--546, the Plague of Justinian is assumed to have killed up to half of the population of affected areas. The effect on cities was disproportionately severe. In Constantinople, for example, an estimated 230'000 people out of 375'000 lost their lives to the disease \citep{Treadgold1997}. Returning in the years 1347--1351, known today as the Black Death, a plague pandemic again wiped out around half of Europe's population. Death toll estimates range from 15 to 23.5 million \citep{Zietz2004}. Leaving behind a grim cultural heritage, this catastrophe had a lasting effect on economic and social structures in Europe. The third large-scale outbreak started around 1855 in southern China and quickly spread to Japan, Taiwan and India again wreaking havoc on the affected population.

Bringing diseases such as smallpox, measles (an infection with the \textit{Measles virus}) and typhus to the Americas during the European invasion of the New World had grave repercussions for the indigenous population, carrying no natural resistance towards the newly introduced pathogens. It is estimated that the population of present day Mexico fell from 20 million to 1.6 million over the course of the 16th century due to multiple disease epidemics, critically contributing to the successful colonization of the new continents \citep{Dobson1996}.

Cholera and influenza are further contagious diseases with high mortality rates, responsible for global epidemics. \textit{Vibrio cholerae}, a bacterium which causes infections of the intestine, became widespread in the early 19th century and caused seven pandemics since, the last of which only started in 1961. Antibacterial treatment of sewage and purification of drinking water greatly help to prevent and contain spreading of the disease but in areas with inadequate sanitation, such as Haiti after the 2010 earthquake, it remains a pathogen difficult to control. The influenza virus causes seasonal epidemics characterized by low lethality rates among people with intact immune systems\footnote{In spite of low lethality, these seasonal epidemics still incur significant economic damages. The \cite{WHO2003} estimates annual health care costs and loss of productivity due to influenza at US \$71--176 billion for the United States of America alone.}. Irregularly occurring influenza pandemics, initiated by zoonosis of new virus strains, against which no natural immunity exists, however, are accompanied by much higher lethality rates. The most significant such event is known today as the Spanish flu pandemic of 1918, costing the lives of 50--100 million, nearly half of which were young, healthy adults \citep{Taubenberger2006}.

With better knowledge of these diseases, effective countermeasures could be developed. Identifying vectors and natural reservoirs, as well as understanding how transmission between infected individuals occur greatly helps to stymie burgeoning outbreaks of infective pathogens and prevent their spreading. In the case of plague, insecticides killing fleas were successfully used as a prophylactic measure, as well as controlling rat populations. Improvements in sanitary conditions and general population health are further important contributing factors to the decline of certain infectious diseases. Most important of all are advancements in medicine such as the development of vaccines and antibiotics. Among the great successes of widespread vaccination efforts is the global eradication of smallpox through a coordinated initiative lead by the World Health Organization in the 1970's.

\input{R/who-deaths/cleanData}
\begin{knitrout}
\definecolor{shadecolor}{rgb}{0.969, 0.969, 0.969}\color{fgcolor}\begin{figure}
\includegraphics[width=\maxwidth]{figures/R/who-deaths/topCauses-who-deaths_top-causes-1} \caption[Relative frequencies of death causes in 2012 by World Bank income groups]{Relative frequencies of death causes in 2012 by World Bank income groups. Binning is based on Gross National Income (GNI) per capita and the thresholds are \$1045 or less for low income, \$1046 to \$4125 for lower-middle, \$4126 to \$12745 for upper-middle and \$12746 or more for high income economies. The data was obtained from the \cite{WHO2012}.}\label{fig:who-deaths_top-causes}
\end{figure}


\end{knitrout}

\newcommand{\knitrTotalDeathsTwelve}{58.3 million}

\newcommand{\knitrPercentageDeathsTwelveHigh}{20.1\%}
\newcommand{\knitrPercentageDeathsTwelveLow}{14\%}
\newcommand{\knitrPercentageDeathsTwelveLmid}{36.5\%}
\newcommand{\knitrPercentageDeathsTwelveUmid}{29.4\%}

\newcommand{\knitrPercentDeathsTwelveLowInfect}{39.6\%}
\newcommand{\knitrPercentDeathsTwelveLowPerinat}{20.8\%}
\newcommand{\knitrPercentDeathsTwelveLmidInfect}{23.3\%}
\newcommand{\knitrPercentDeathsTwelveLmidCardio}{26.5\%}
\newcommand{\knitrPercentDeathsTwelveUmidInfect}{8.5\%}
\newcommand{\knitrPercentDeathsTwelveHighInfect}{6.7\%}
\newcommand{\knitrPercentDeathsTwelveWorldInfect}{18.3\%}
\newcommand{\knitrPercentDeathsTwelveWorldCardio}{33.7\%}
\newcommand{\knitrPercentDeathsTwelveWorldCancer}{15.8\%}


Despite development of means to treat and prevent many previously devastating diseases, infectious pathogens remain a serious threat to global health. In 2012, an estimated total of \knitrTotalDeathsTwelve{} people died (\knitrPercentageDeathsTwelveHigh{} in high, \knitrPercentageDeathsTwelveUmid{} in upper-middle, \knitrPercentageDeathsTwelveLmid{} in lower-middle and \knitrPercentageDeathsTwelveLow{} in low income countries). Figure \ref{fig:who-deaths_top-causes} partitions the total death count into World Bank income groups and causes. In low income countries, infective diseases are the most prevalent cause of death (\knitrPercentDeathsTwelveLowInfect{}), followed by maternal and perinatal complications with substantial margin (\knitrPercentDeathsTwelveLowPerinat{}). In lower middle income countries, cardiovascular conditions catch up (\knitrPercentDeathsTwelveLmidCardio{}), but are still almost matched in frequency by infectious diseases (\knitrPercentDeathsTwelveLmidInfect{}). In upper middle (\knitrPercentDeathsTwelveUmidInfect{}) and high income countries (\knitrPercentDeathsTwelveHighInfect{}), the importance of infectious disease while weakened remains accountable for a significant number of deaths. Globally, infectious diseases are the second most frequent cause of death (\knitrPercentDeathsTwelveWorldInfect{}), only preceded by cardiovascular diseases (\knitrPercentDeathsTwelveWorldCardio{}).

\begin{knitrout}
\definecolor{shadecolor}{rgb}{0.969, 0.969, 0.969}\color{fgcolor}\begin{figure}
\includegraphics[width=\maxwidth]{figures/R/who-deaths/byDisease-who-deaths_by-disease-1} \caption[Relative frequencies deadly infectious diseases for 2012 by World Bank income groups]{Relative frequencies deadly infectious diseases for 2012 by World Bank income groups. Binning is based on Gross National Income (GNI; see figure \ref{fig:who-deaths_top-causes}). The data was obtained from the \cite{WHO2012}.}\label{fig:who-deaths_by-disease}
\end{figure}


\end{knitrout}

\newcommand{\knitrPercentageInfectTwelveWorldLRI}{34.5\%}
\newcommand{\knitrPercentageInfectTwelveHighLRI}{57.7\%}
\newcommand{\knitrPercentageInfectTwelveUmidLRI}{43.5\%}
\newcommand{\knitrPercentageInfectTwelveLmidLRI}{30.8\%}
\newcommand{\knitrPercentageInfectTwelveLowLRI}{28.7\%}
\newcommand{\knitrPercentageInfectTwelveHighDiarr}{5.6\%}
\newcommand{\knitrPercentageInfectTwelveUmidDiarr}{7\%}
\newcommand{\knitrPercentageInfectTwelveLmidDiarr}{21.4\%}
\newcommand{\knitrPercentageInfectTwelveLowDiarr}{16.6\%}
\newcommand{\knitrPercentageInfectTwelveWorldAIDS}{17.3\%}
\newcommand{\knitrPercentageInfectTwelveWorldDiarr}{16.9\%}
\newcommand{\knitrPercentageInfectTwelveHighAIDS}{11.3\%}
\newcommand{\knitrPercentageInfectTwelveUmidAIDS}{26.2\%}
\newcommand{\knitrPercentageInfectTwelveLmidAIDS}{13.3\%}
\newcommand{\knitrPercentageInfectTwelveLowAIDS}{20.4\%}


Focusing only on deaths caused by infectious disease, lower respiratory infections are most frequent (for each income region individually, low to high: \knitrPercentageInfectTwelveLowLRI{}, \knitrPercentageInfectTwelveLmidLRI{}, \knitrPercentageInfectTwelveUmidLRI{} and \knitrPercentageInfectTwelveHighLRI{} as well as worldwide: \knitrPercentageInfectTwelveWorldLRI{}; cf. figure \ref{fig:who-deaths_by-disease}). Diarrhoeal diseases and HIV/AIDS are the next most common worldwide (\knitrPercentageInfectTwelveWorldDiarr{} and \knitrPercentageInfectTwelveWorldAIDS{}, respectively) where diarrhea is more prevalent in lower income regions (\knitrPercentageInfectTwelveLowDiarr{} and \knitrPercentageInfectTwelveLmidDiarr{} versus \knitrPercentageInfectTwelveUmidDiarr{} and \knitrPercentageInfectTwelveHighDiarr{}), while HIV/AIDS plays a major role irrespective of income region (low to high: \knitrPercentageInfectTwelveLowAIDS{}, \knitrPercentageInfectTwelveLmidAIDS{}, \knitrPercentageInfectTwelveUmidAIDS{} and \knitrPercentageInfectTwelveHighAIDS{}).


\include{chapters/rules}
\include{chapters/typography}
\chapter{First Chapter SfS Template} 

\section{To include a picture}
%% picture 'h'ere, 'b'ottom or 't'op; '!' try to impose your will on latex
\begin{figure}[hbt!]
  %% no file extension; .85 stands for 85% of text width
  \epsfCfile{.85}{geys-2kern} 
  %% legend for the list of figures at the beginning of you thesis
  \caption[Geyser data: binned histogram, Silverman's and another kernel]
  %% legend displayed below the graph.
  {Old Faithful Geyser eruption lengths, $n=272$; binned data and two
    (Gaussian) kernel density estimates ($\times 10$) with $h=h^*= .3348$
    and $h= .1$ (dotted).}
  \label{fig:geys1}
\end{figure}

Or also with \texttt{includegraphics}:
%% picture 'h'ere, 'b'ottom or 't'op; '!' try to impose your will on latex
\begin{figure}[hbt!]
  \centering
  %% no file extension; .5\textwidth stands for 50% of text width
  \includegraphics[width=.5\textwidth]{geys-2kern}
  %% legend for the list of figures at the beginning of you thesis
  \caption[Geyser data: binned histogram, Silverman's and another
  kernel]
  %% legend displayed below the graph.
  {Old Faithful Geyser eruption lengths, $n=272$; binned data and two
    (Gaussian) kernel density estimates ($\times 10$) with $h=h^*= .3348$
    and $h= .1$ (dotted).}
  \label{fig:geys2}
\end{figure}

\section{To make a proof}
\begin{proof}
  $1 + 1 = 2$
\end{proof}

\section{To include \Rp code}
See information in Appendix~\ref{app:complement}.


\section{Other information}
Put a text between quotes: make sure to use nice quotes, such as ``quote''.

Cite a document in the bibliography (an example here): \cite{Gelman2008}.
Or mention that \citeauthor{Konis2007} (a person) or \citeauthor{Hastie2009} (multiple
persons) have already done quite a bit work.

Referencing a different part of your work: please refer to Appendix \ref{app:complement}.


\appendix

\include{chapters/appendix}

\backmatter

\bibliographystyle{plain}
\bibliography{refs}

\includepdf[pages={-}]{declaration-originality.pdf}

\end{document}
